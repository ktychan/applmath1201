%! TeX program = lualatex
\documentclass[../main.tex]{subfiles}
\begin{document} \section{Developing continuous-time models using differential equations}

Before we investigate mathematical properties of differential equations, let's see why we should care about their properties at all by doing some real-world modelling. 

\begin{method}
  Use differential equations to describe the rates of change of physical \hlmain{systems}.  The simplest idea is
  \begin{center}
    (rate of change) = (rate of internal change) + (rate of external change).
  \end{center}
\end{method}
\blanklines{5}

The phrase ``\emph{rate of change is \hlattn{proportional} to}'' is a good starting point to discuss rate of \hlmain{internal} change for many biological systems.  That \hlmain{the rate of change (with respect to time \(t\)) of some quantity \(P\)} is \hlattn{proportional} to \hlsupp{\(P\)} translates to a differential equation 
\[
  {\color{main}P'(t)} = {\color{attn} k} {\color{supp} P(t)},\quad \text{ for some constant } k.
\]

Assume \(P \ge 0\).  The sign of \(k\) reveals qualitative information.  Use our calculus knowledge to complete the table.

\begin{table}[H]
  \centering
  \begin{tabular}{r|p{1.5in}|p{1.5in}|p{1.5in}}
  & \(k < 0\) & \(k = 0\) & \(k > 0\) \\ \midrule
    how \(P\) changes as \(t\) increases & & & \\[2ex]
  \end{tabular}
\end{table}

For the next four examples, we consider the interaction between a made-up anaesthesia drug ABC and the human bloodstream. Eventually, we would like to answer the question ``how much ABC will keep a patient under for \emph{blah} minutes?''

Assume that the human body does not naturally produce ABC, and ABC is not present in the human body unless it is administered externally. 

Assume the amount of ABC is measured in \emph{mg} and time in \emph{minutes}.

\begin{example}[modelling with only rate of internal change] \label{ex:diff-eq-ABCv1}
  Experimental data suggests that the \hlmain{rate of change of ABC} is \hlattn{proportional} to \hlmain{the amount of ABC present}, decaying by a factor of \(1\%\) per minute.

  Write down a \emph{continuous-time model} of ABC present in the bloodstream.
  \blanklines{10}
\end{example}

\begin{example}[modelling with rates of internal and external change] \label{ex:diff-eq-ABCv2}
  Continue from Example~\ref{ex:diff-eq-ABCv2}.  Suppose further \(ABC\) is \emph{continuously} administered at a rate of \(r\) mg per minute. 

  Write down a \emph{continuous-time model} of ABC in the bloodstream.
  \blanklines{15}
\end{example}

\begin{example} \label{ex:diff-eq-ABCv3}
  Continue from Example~\ref{ex:diff-eq-ABCv2}.  Suppose further that \(ABC\) exits the bloodstream through blood loss at a rate of \(s\) mg (of ABC) per minutes. 

  Write down a \emph{continuous-time model} of ABC in the bloodstream.
  \blanklines{15}
\end{example}

\begin{example}[sanity check]
  Verify that the model we developed in Example~\ref{ex:diff-eq-ABCv3} is dimensionally homogeneous.
  \blanklines{10}
\end{example}

We will revisit (a new and improved) ABC to solve problems using differential equations.
\clearpage

\begin{example}
  A bathtub is filled with \(V\) litres of hot water. The rate of water \emph{evaporation} is proportional to the volume of water by a factor of \(0.1\) per minute.  Water (of the same temperature) enters the tub at a rate of \(1\) litre per minute. The tub leaks \(0.1\) litre per minute. 

  Model the volume of water in the tub with respect to time \(t\) using a differential equation.

  Hint: What information is given? Rate of growth or rate of decay?
  \blanklines{15}
\end{example}
\end{document}
