%! TeX program = lualatex
\documentclass[../main.tex]{subfiles}
\begin{document} \section{Applications of differential equations}
We apply the idea in Method~\ref{method:diff-eq-model} and to model more complex behaviour. 

The problems in this section differ from Examples~\ref{ex:diff-eq-ABCv1}, \ref{ex:diff-eq-ABCv2} and \ref{ex:diff-eq-ABCv3} in that the quantity entering or exiting the system is no longer a constant. However, the idea stays the same. We use given information to describe internal and external rates of change of \hlmain{the desired quantity}.

\faStar{} Remember, ensuring all units match often keeps us on the right track.

\begin{example}
  A tube contains \(1\) litre of clear liquid. Starting at time \(t = 0\), a green liquid with concentration \(10\) grams per litre of green particles flows into the one side of the tube continuously at a rate of \(0.1\) litres per minute.  The mixture exits the other end of the tube at a rate of \(0.1\) litres per minute.

  To give ourselves a sufficiently simple relation \((\text{mass}) = (\text{concentration \emph{constant}}) \times (\text{volume})\), we assume that the liquid inside the tube is \emph{well-mixed} at all times so that the concentration of green particles is \emph{constant} everywhere. Realistic? Depends. Reasonable? Sometimes. Computable? Yes!

  Let \(G(t)\) be the mass of green particles inside the tube at time \(t\). Develop an initial-value problem to which \(G(t)\) is the solution.
  
  \blanklines{35}
\end{example}
\clearpage

\begin{example} \label{ex:diff-eq-model-mixing-1}
  A saline solution is a mixture of sodium chloride and water. 

  A tank contains \(10\) L of water with no sodium chloride. At time \(t = 0\), a saline solution with concentration \(\tfrac{1}{4}\) mg per mL flows continuously into the tank at a rate of \(10\) mL per second. The tank leaks at a rate of \(10\) mL per second. Assume there is no loss of water due to evaporation.

  Assume the tank is \emph{well-mixed} at all times, so that the concentration of sodium chloride and water is constant everywhere. 

  Let \(u(t)\) be the mass of sodium chloride in the tank at time \(t\).  Develop an initial-value problem to which \(u(t)\) is the solution. 

  \blanklines{45}
\end{example}
\clearpage

\begin{example}[modelling with loss] \label{ex:diff-eq-model-mixing-2}
  Continue from Example~\ref{ex:diff-eq-model-mixing-1}. Now assume further the saline solution is hot and \emph{pure} water evaporates at a rate of \(0.01\) mL per second.  Assume everything else stays the same.

  Evaporation means \emph{pure} water leaves the tank and \emph{does not} change in the mass of sodium chloride in the tank.

  Propose a model for the mass of sodium chloride in the tank at time \(t\). 
  \blanklines{45}
\end{example}
\clearpage

\begin{example}[modelling with feedback] \label{ex:diff-eq-model-mixing-3}
  Continue from Example~\ref{ex:diff-eq-model-mixing-1}. Assume further that \(1/3\) of saline solution exiting the tank is fed back into the tank. 

  Assume there is no evaporation.

  Propose a model for the amount of sodium chloride in the tank at time \(t\). 
  \blanklines{45}
\end{example}
\clearpage

\begin{example}
  Continue from Example~\ref{ex:diff-eq-model-mixing-1}. Assume instead that initially, the tank contains \(10\) litres of saline solution at \(0.1\%\) concentration.

  Propose a model for the amount of sodium chloride in the tank at time \(t\). 
  \blanklines{50}
\end{example}
\end{document}
