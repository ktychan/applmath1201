%! TeX program = lualatex
\documentclass[../main.tex]{subfiles}
\begin{document} \section{Strategies and more exercises}

We discuss problem-solving strategies.  
\blanklines{50}
\clearpage

\begin{example}[\hlmain{organize knowledge by thinking patterns, not wording}] \label{ex:strategy-connection}
  Compare the problems below.

  From quiz 1 ...
  \begin{quote}
    An initial investment of \(3,000\) dollars accrues annually compounded interest at a rate of \(15\%\). At the end of every year the investor can withdraw some fixed dollar amount from the investment. What fixed annual withdrawal would leave the investment worthless after exactly \(15\) years?
  \end{quote}

  A slightly modification of Example~2 on page 33 of the textbook.
  \begin{quote}
    A particular drug is metabolized at a rate of \(10\%\) per hour. Assume the drug is administered at one point in time only, how long will it take for the amount in a patient's body to fall below \(1\%\) of the original dose?

    Metabolism yields an internal rate of change.
  \end{quote}

  Surprisingly (or maybe not) the same problem-solving strategy works for both problems.  To see this, we throw away details of calculations.

  \blanklines{35}
\end{example}
\clearpage

\begin{example}[\hlmain{follow the method, not the numbers}]
  Compare the following modelling problems. 

  Example~\ref{ex:diff-eq-model-mixing-1}.
  \begin{quote}
    A well-mixed tank contains \(10\) L of water with no sodium chloride. At \(t = 0\), a saline solution with concentration \(1/4\) mg per mL flows continuously into the tank at a rate of \(10\) mL per second. The tank leaks at a rate of \(10\) mL per second. 

    Let \(u(t)\) be the mass of sodium chloride in the tank at time \(t\). Develop an initial value problem to which \(u(t)\) is the solution.
  \end{quote}
  \blanklines{10}

  A variant of Example~\ref{ex:diff-eq-model-mixing-1}.
  \begin{quote}
    A well-mixed tank contains \(10\) L of water with no sodium chloride. At \(t = 0\), a saline solution with concentration \(1/4\) mg per mL flows continuously into the tank at a rate of \(10\) mL per second. The tank leaks at a rate of \(15\) mL per second. 

    Let \(u(t)\) be the mass of sodium chloride in the tank at time \(t\). Develop an initial value problem to which \(u(t)\) is the solution.
  \end{quote}
  \blanklines{25}
\end{example}
\clearpage

Let's reinforce a sub-strategy often used in Calculus~1000.  In fact, we used this in Example~\ref{ex:strategy-connection}. 

\begin{example}
  A function \(u(t) = e^{kt^{2}} - 1\) is a solution to the differential equation \(u'(t) - tu(t) = t^{2}\).  Find the unknown constant \(k\).

  \blanklines{20}
\end{example}

\clearpage
Here are some exercises for scientific computing.
\begin{example}
  What is the output of the following Python code?

  \begin{pythoncode}
print("The first item in an list called a is a[0].")
  \end{pythoncode}
\end{example}

\begin{example}
  What is the output of the following code?

  \begin{pythoncode}
a = [1, 1, 8, 5]
print( a[2] * 5 )
  \end{pythoncode}
\end{example}


\begin{example}
  What is the output of the following code?
  \begin{pythoncode}
import numpy

a = [1, 1, 8, 5]
print( numpy.log(a) )
  \end{pythoncode}
\end{example}

\begin{example}
  What is the output of the following code?

  \begin{pythoncode}
import numpy

numpy.arange(-1, 15.1, 4)
  \end{pythoncode}
\end{example}

\begin{example}
  What points do you expect to see on the output of the following code?

  \begin{pythoncode}
import matplotlib.pyplot as plt
import numpy

plt.xlim(0, 5);
plt.ylim(0, 5);
plt.xticks(numpy.arange(0, 5.1, 1))
plt.yticks(numpy.arange(0, 5.1, 1))
plt.grid(linestyle = '-.', linewidth = 0.5)

a = [1, 2]
b = [3, 4]
plt.plot(a, b, 'o');
  \end{pythoncode}
\end{example}

\end{document}
