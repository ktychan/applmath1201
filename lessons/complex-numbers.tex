%! TeX program = lualatex
\documentclass[../main.tex]{subfiles}
\begin{document} \section{Complex numbers}

Complex numbers come from the theoretical need to solve equations such as \(x^{2} = -1\) but are fundamentally useful in all sciences.  In this section, we learn complex numbers for the purpose of understanding how vectors move in a vector field by calculating eigenvalues in the next section.

\begin{definition}[complex numbers]
  The symbol \(i\) is called \hlmain{\emph{the} imaginary number}, and define
  % is forced to satisfy
  \[
    \sqrt{-1} = \underline{\hspace{1cm}}
    \quad\text{and}\quad 
    i^{2} = \underline{\hspace{1cm}}.
  \]

  \hlmain{Complex numbers} are \hlsupp{polynomials in \(i\)}, but powers of \(i\) \hlwarn{must be simplified}.
  % polynomials in i

  A complex number can always be simplified to \underline{\hspace{2cm}} where \underline{\hspace{1.5in}}. In this simplified form, its \hlmain{real part} is \underline{\hspace{1cm}}, and \hlmain{imaginary part} is \underline{\hspace{1cm}}.
\end{definition}
\faStar{} Think about \(i\) as a symbolic shorthand for \(\sqrt{-1}\) and use your high school algebra intuition.


\begin{example}
  Identify the real and imaginary parts of each complex number below. 

  \begin{itemize}[wide]
    \item \(-\pi + 3i - 5i + \sqrt{-2}\)
      \blanklines{12}

    \item \(-5 + i^{30} \)
      \blanklines{12}
      \clearpage

    \item \((3 + 5i) (4 - 6i^{3})\)
      \blanklines{12}

    \item \(\frac{1 + 2i}{1 + 2i + \sqrt{-4}}\)
      \blanklines{12}
  \end{itemize}
\end{example}

For any complex number \underline{\hspace{1in}} to the form \(z = a + bi\) where \(a,b\) are \underline{\hspace{3cm}}, define its \hlmain{complex conjugate} to be \(\bar{z} = \underline{\hspace{1in}}\) and its \hlmain{length} to be \(|z| = \underline{\hspace{3cm}}\).  It follows that division has a shorter formula
\[
  \frac{1}{z} = \hspace{1in}
\]
\blanklines{5}

\faStar{} We use complex numbers and the quadratic formula to solve equations appearing later this week.
\begin{example}
  Let \(\lambda\) be an unknown. Solve \((\lambda - 5) (\lambda^{2} + 5) = 0\).
  \blanklines{10}
\end{example}

\begin{example}
  Let \(\lambda\) be an unknown. Solve \(\lambda^{2} + 3\lambda + 1 = 0\).
  \blanklines{15}
\end{example}

\begin{example}
  Let \(\lambda\) be an unknown. Solve \((\lambda - 10) (\lambda^{2} + 5\lambda - 7) = 0\).
  \blanklines{15}
\end{example}

We can visualize algebraic operations for complex number because they can always be simplified to the form \(a + bi\) where \(a,b\) are real numbers.

\begin{example}
  Visualize \((2 + 3i) + (-1 + i)\) and \(i (4 + i)\).

  \includegraphics{../standalones/build/plot-complex-plane}
\end{example}

\end{document}
