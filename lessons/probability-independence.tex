%! TeX program = lualatex
\documentclass[../main.tex]{subfiles}
\begin{document} \section{The law of total probability and independence}

The law of total probability mathematically takes advantages of mutually exclusive and exhaustive events to compute probabilities.

First, let's express \(\mathbb{P}(E \cap F)\) in terms of \(\mathbb{P}(E \mid F)\).
\blanklines{5}

If we divide the sample space into mutually exclusive and exhaustive events \(F_{1}\) and \(F_{2}\) and try to compute \(\mathbb{P}(E)\), then we should get ...
\blanklines{10}


If we divide the sample space into mutually exclusive and exhaustive events \(F_{1}, F_{2}, F_{3}\) and try to compute \(\mathbb{P}(E)\), then we should get ...
\blanklines{10}

\faStar{} The law of total probability is the above pattern in full generality. Suppose \(B_{1}, \ldots, B_{n}\) are mutually exclusive and exhaustive events in a sample space \(\Omega\). The \hlmain{law of total probability} is a way to express \(\mathbb{P}(E)\) as ...
\blanklines{10}
\clearpage

Intuitively, two events are independent if the occurrence of one does not affect the other.

\begin{definition}[independent events]
  Two events \(E,F\) in a common sample space are said to be \hlmain{independent} if 
  \begin{equation} 
    \mathbb{P}(E \cap F) = \mathbb{P}(E) \mathbb{P}(F). 
  \end{equation}
\end{definition}

\faStar{} Events \(E,F\) are independent events if and only if
\begin{equation} \label{eq:probability-conditional-of-independent}
  \mathbb{P}(E \mid F) = \hspace{1.5in}
  \quad\text{ and }\quad
  \mathbb{P}(F \mid E) = \hspace{1.5in}
\end{equation}
\blanklines{5}

The above offers several ways to identify independent events.
\blanklines{10}

\begin{example}
  Let \(E, F\) be events in a common sample space. Let \(\mathbb{P}(E) = 0.3\), \(\mathbb{P}(B) = 0.2\) and \(\mathbb{P}(A \cap B) = 0.06\). Determine if \(E,F\) are independent.
  \blanklines{5}
\end{example}

\begin{example} 
  Let \(E,F\) be two events in a sample space \(\Omega\). If \(E,F\) are independent events and \(\mathbb{P}(E) > 0\), does that imply \(E\) and \(F\) are mutually exclusive?

  \blanklines{10} 
\end{example}

\begin{example}
  If \(E\) and \(F\) are independent events, does that mean \(E^{c}\) and \(F\) are also independent events?

  \blanklines{5}
\end{example}

\end{document}
