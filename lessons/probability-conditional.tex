%! TeX program = lualatex
\documentclass[../main.tex]{subfiles}
\begin{document} \section{Conditional Probability}

\begin{objective}
  \begin{enumerate}
    \item calculate the probability of an event \(E\) given that another event \(F\) has already occurred,
    \item apply the \hlmain{weak law of large numbers},
    \item understand \hlmain{mutually exclusive and exhaustive} events,
    \item apply \hlmain{the law of total probability}
    \item understand \hlmain{conditional independence} 
  \end{enumerate}
\end{objective}

Let \(E,F\) be two events in a sample space \(\Omega\). The probability of \hlmain{\(F\) occurring given \(E\)} has already occurred is defined to be
\[
  \mathbb{P}(F \mid E) = \frac{\mathbb{P}(F \cap E)}{\mathbb{P}(E)}.
\]

\begin{example}
  Let \(E,F\) be two events in a sample space \(\Omega\). If \(E,F\) are independent events, does that mean \(E\) and \(F\) are mutually exclusive?
\end{example}
\end{document}
