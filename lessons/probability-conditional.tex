%! TeX program = lualatex
\documentclass[../main.tex]{subfiles}
\begin{document} \section{Conditional Probability}

We now consider probability of an event \hlmain{given} that another event has already happened.   We consider a concrete example from the first principle to motivate the formula associated to conditional probability.

\begin{example} \label{ex:probability-conditional-motivation}
  Let \(\Omega = \{1,2,3,4,5,6\}\) be the sample space of six people.  It is known that 
  \begin{itemize}
    \item \(1,2,3,4\) like ketchup flavoured chips,
    \item \(5,6\) like barbecue flavoured chips,
    \item \(1,3,5\) like coffee, and
    \item \(2,4,6\) like tea.
  \end{itemize}

  What is the probability of \emph{randomly} selecting someone who like coffee \emph{given} that we know they like ketchup chips?

  \blanklines{15}
\end{example}

\begin{definition}[conditional probability]
  Suppose \(E,F\) are events inside a common sample space \(\Omega\).  The probability of \hlmain{\(E\) given \(F\)} is the probability of \(E \cap F\) inside the smaller sample space \(F\) and
  \[
    \mathbb{P}( E \mid F ) = \frac{\mathbb{P}(E \cap F)}{\mathbb{P}(F)}.
  \]

  The probability \(\mathbb{P}( E \mid F )\) is called a \hlmain{conditional probability} because it represents the probability of the event \(E\) under the condition that \(F\) has already happened.
\end{definition}

\begin{example}
  Express the probability in Example~\ref{ex:probability-conditional-motivation} in symbols.
  \blanklines{5}
\end{example}
\clearpage


\begin{example}
  Suppose the probability of having green eyes is \(10\%\), the probability of having brown hair is \(75\%\), and the probability of having both green eyes and brown hair is \(9\%\), what is the probability of having brown hair given that you have green eyes?
  
  {\footnotesize Reference. Exercise P3.2 of \emph{A Biologist's Guide to Mathematical Modeling in Ecology and Evolution} by Otto and Day.}

  \blanklines{5}
\end{example}

It takes a little bit of practice to detect conditional probability from natural language.
\begin{example}
  A ball pit contains balls labelled \(1\) and \(2\), some are black and some are white.  \(13\%\) of the balls are black and labelled \(1\), \(25\%\) black and labelled \(2\), \(10\%\) white and labelled \(1\), and the rest are white and labelled \(2\).

  What is the probability of randomly picking a white one from the balls labelled \(2\)?
  \blanklines{5}
\end{example}

\begin{example}
  A clothing store is stocked with \(95\) T-shirts, \(25\) jackets, \(15\) dress pants, \(40\) sweat pants and \(100\) socks. What is the probability that a randomly selected pair of pants is a pair of sweat pants?
  \blanklines{10}
\end{example}

\begin{example}
  Suppose \(\mathbb{P}(E) = 0.2\), \(\mathbb{P}(F) = 0.5\) and \(\mathbb{P}(E \mid F) = 0.25\). Calculate \(\mathbb{P}(E \cap F)\).
  \blanklines{5}
\end{example}

\begin{example}
  Suppose \(\mathbb{P}(E \cup F) = 0.9\), \(\mathbb{P}(E) = 0.7\), \(\mathbb{P}(E \cap F) = 0.5\). Calculate \(\mathbb{P}(E \mid F)\).
  \blanklines{5}
\end{example}

\end{document}
