%! TeX program = lualatex
\documentclass[../main.tex]{subfiles}
\begin{document} \section{Introduction to random variables}

In a nutshell, a \hlmain{random variable} is a mathematically computable formality to \hlsupp{represent a two-column table} (could have finitely or infinitely many rows) associating each outcome in a sample space \emph{to} a number.  Random variables are statistical models.

We build some intuition through an example before thinking about formalities. 

\begin{example}[height distribution is a random variable] \label{ex:random-variable-intro}
  Consider the following questions related to the height data of some mythical figures.

  Pay attention to how we use the data to \hlattn{formulate} each question in the language of probability, i.e., sample space and events.

  {\color{main}
    What is the probability of randomly selecting someone whose height is 
    \begin{enumerate}
      \item exactly \(170\) cm?
      \item between \(160\) and \(180\) cm (inclusive)?
      \item at least \(175\) cm?
    \end{enumerate}
  }

  \hfill{}
  \begin{tabular}{l|l}
    Person & Height \\\midrule
    Zeus & \(165\) \\
    Hera & \(155\) \\
    Perseus & \(180\) \\
    Theseus & \(160\) \\
    Athena & \(170\) \\
    Poseidon  & \(170\) \\
    Hades & \(160\) \\
    Helios & \(170\) \\
    Selene & \(180\) \\
    Achilles & \(180\)
  \end{tabular}

  \blanklines{20}
\end{example}
\clearpage

Formally, a \emph{random variable}, typically denoted by a capital letter \(X\), is a \emph{function} whose domain is a sample space and whose range is either a set of discrete numbers, e.g., \(\{1,2,3,\ldots\}\), or an interval. 

We need to correctly interpret various notations associated to random variables.
\begin{definition} \label{def:random-variables}
  A random variable \(X\) on \(\Omega\) associates outcomes to values.

  \begin{itemize}[itemsep={2ex}]
    \item The notation 
      \[
        X(\text{the name of an outcome})
      \] 
      represents the value associated to the named outcome.
      
      % For example, if \(X\) is a random variable recording heights of everybody in our class, then \(X(\underline{\hspace{1in}})\) is \underline{\hspace{1.5cm}}.

    \item The notation 
      \[
        X = (\text{some value})
      \] 
      represents \underline{\hspace{2in}} containing outcomes whose \underline{\hspace{2cm}} is equal to the \emph{given} value on the right-hand side. 
      Consequently, 
      \[
        \mathbb{P}(X = (\text{some value}))
      \]
      is \underline{\hspace{5in}} % probability of the event X = (some value).

    \item The notation (and its variants involving different inequalities)
      \[
        a \le X \le b
      \]
      represents \underline{\hspace{2in}} containing outcomes whose \underline{\hspace{2cm}} satisfies \underline{\hspace{4in}}.
      Consequently,
      \[
        \mathbb{P}(a \le X \le b)
      \]
      is \underline{\hspace{5in}} % probability of the event a ≤ X ≤ b.
  \end{itemize}
\end{definition}

\begin{example}
  Let \(X\) be a random variable representing heights of the class.  Match notations to their descriptions. 

  \begin{multicols}{2}
   \begin{itemize}
      \item Everyone whose height is in \((160, 180]\).
      \item Everyone who is at most \(180\) cm tall.
      \item Someone who is exactly \(180\) cm tall.
      \item Everyone who is exactly \(180\) cm tall.
      \item Alex is \(180\) cm tall.
      \item The height of Alex.
      \item The entire sample space.
      \item Does not make sense. 
    \end{itemize} 
    \columnbreak
    \begin{itemize}
      \item \(X \le 180\)
      \item \(160 < X \le 180\)
      \item \(X = 180\)
      \item \(X(\text{Alex})\)
      \item \(X(180)\)
      \item \( X < \infty\)
    \end{itemize}
  \end{multicols}
\end{example}
\end{document}
