%! TeX program = lualatex
\documentclass[../main.tex]{subfiles}
\begin{document} \section{Random variables}

We can introduce random variables as data collection devices.

Consider a certain population. Let \(X_{1}\) record heights and \(X_{2}\) weights. More precisely, if \(i\) is someone in the population, then \(X_{1}(i)\) is the height of \(i\), and \(X_{2}(i)\) is the weight of \(i\).

Mathematically, think of them as piecewise functions.

\begin{definition}[probability mass function]
  The probability density function \(f(x)\) for a discrete random variable \(X\) is defined by
  \[
    f(x) = \mathbb{P}(X = x),
  \]
  where \(x\) is any outcome of \(X\).
\end{definition}

\begin{definition}[probability density function]
  The probability density function \(f(x)\) for a continuous random variable \(X\) is a function \(f(x)\) so that
  \[
    \mathbb{P}(a \le X \le b) = \int_{a}^{b} f(x) \;dx.
  \]
\end{definition}

\begin{definition}[cumulative distribution function]
  A function \(f(x)\) defined on all real numbers is called a probability density function if \(f\) is non-negative and 
  \[
    \int_{-\infty}^{\infty} f(x) \;dx = 1.
  \]

  Given a continuous random variable \(X\), we can define a continuous random variable \(X\) by imposing that
  \[
    
  \]
\end{definition}

\end{document}
