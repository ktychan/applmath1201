%! TeX program = lualatex
\documentclass[../main.tex]{subfiles}
\begin{document} \section{Age-structured population}

We can use vector and matricies multi-dimensional phenomenon, such as age-structured population.  By age-structured population, we mean a population with two mutually exclusive subgroups \emph{adults} and \emph{juveniles} and consider the effect of growing up.

\begin{example}[Example~6 on page 157 of the textbook]
  Consider a population of juveniles and adults. Initially, there are \(20\) juveniles and \(40\) adults. At the end of each year, 
  \begin{itemize}[itemsep={0ex}]
    \item \(20\%\) of the juveniles become adults, and the rest juveniles die, 
    \item \(90\%\) of the adults survive, and the rest of adults die, and
    \item only adults produce offsprings, and the per-capita birth rate is \(15\%\).
  \end{itemize}

  Predict the number of adults and juveniles after \(2\) years.  How would you predict the populations at the end of \(n\) years?
  \blanklines{35}

  \clearpage
  \blanklines{50}
\end{example}
\clearpage

\begin{example}
  Write down the matrix representing the following network.

  \begin{center}
    \includegraphics{../standalones/build/diagram_transition.pdf}
  \end{center}

  Moreover, describe the transition between juveniles and adults in words.

  \blanklines{20}
\end{example}

\begin{example}
  Express the recurrence equation \(\vec{x}(t+1) = A \vec{x}(t)\) as a network diagram where \(A = \begin{bmatrix} 0.05 & 0.33 \\ 0.2 & 0.5 \end{bmatrix}\). Moreover, describe the transition between juveniles \(x_{1}\) and adults \(x_{2}\) in words.

  \blanklines{20}
\end{example}
\end{document}
