%! TeX program = lualatex
\documentclass[../main.tex]{subfiles}
\begin{document} \section{Developing discrete-time models using recurrences}

Before we investigate mathematical properties of recurrence, let's see why we should care about their properties at all by doing some real-world modelling. 

\begin{method}
  Use recurrence equations to describe how a physical system changes from one state to the next.  Be sure to specify its initial state.
\end{method}


% use the word balance instead of value of the investment
Let's start with an everyday example of discrete-time models.
\begin{example} \label{ex:recurrsion-investment-model}
  We deposit \(1000\) dollars into a savings account. At the end of every year since the initial deposit, interest of \(2\%\) of the current balance is paid. We would like to withdraw \(50\) dollars \emph{after} each interest payment.

  Develop a discrete-time model for the relation between the balance of the saving and time.

  \blanklines{25}
\end{example}

\faComment{} What questions do you have about the investment?
\blanklines{10}

\begin{example} \label{ex:recurrsion-half-life}
  Carbon-14 is a radioactive element that can be used to determine the age of archaeological artifacts of biological origin.  The amount of carbon-14 in a sample decreases by \(50\%\) after roughly \(5730\) years. In other words, the half-life of carbon-14 is \(5730\) years. 

  Develop a discrete-time model for the relation between the amount of carbon-14 in a sample and time.
  \blanklines{30}

  \faComment{} Let's say you detect \(100\) mg of carbon-14 in a sample of dinosaur bones and you wish to determine when the dinosaur lived. \emph{Formulate} this question using the above model. We will fully answer this question later this week.
  \blanklines{15}
\end{example}

Given a recursive discrete-time model, we can find explanation for the recurrence.
\begin{example}
  A house has a cat and some mouse.  The cat chases away a fixed portion of mouse everyday during the day. Voluntary move-ins/outs happen before midnight. 
  A census of the mouse population is taken every midnight.

  An intern proposes the following model for the number of mouse \(m_{t}\) at night \(t\).  
  \[
    m_{0} \text{ is unknown } \quad\text{and}\quad m_{t} = 0.2 m_{t-1} + 5, \text{ for integers } t \ge 1.
  \]

  What percentage of the mouse population is chased away by the cat everyday?
  \blanklines{10}
\end{example}
\end{document}
