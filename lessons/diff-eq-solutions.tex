%! TeX program = lualatex
\documentclass[../main.tex]{subfiles}
\begin{document} \section{Verifying solutions to differential equations}

To verify a proposed solution for a differential equation, we focus on the dependent variable. Here is the general procedure. 

\begin{mdframed}[style=simple-compact]
  Suppose \(y = f(x)\) is a proposed solution for a differential equation in which \(y\) is the dependent variable and \(x\) is the independent variable. 

  \textbf{Preparation}.
  \begin{enumerate}
    \item Reorganize the differential equation to the form \hlmain{\((\cdots \text{stuff} \cdots) = 0\)}.
    \item Discard the ``\hlmain{\(=0\)}'' on the right-hand side. 
  \end{enumerate}

  \textbf{Calculate}.
  \begin{enumerate}
    \item Substitute every \(y\) with \(f(x)\) in \hlmain{\((\cdots \text{stuff} \cdots)\)}. Leave all \(x\)'s unchanged.
    \item Perform all necessary differentiation. 
    \item Simplify.
  \end{enumerate} 

  \textbf{Decide}. If the resulting expression simplifies to \(0\), then \(y = f(x)\) is a solution of the differential equation; otherwise, \(y = f(x)\) is not a solution.
\end{mdframed}

\begin{example}
  Which of the following functions is a solution to the differential equation \(y' = y\)?
  \[
    y = x^{2}, \qquad y = e^{x} + 5, \qquad y = 3e^{x}.
  \]

  \blanklines{20}
\end{example}

\faComment{} Do differential equations have unique solutions?
\clearpage

\begin{example}
  Consider the differential equation \(y' + g(x) y = x\). Define \(u(x) = \int g(x) \;dx\). Is \(y = g(x)u(x)\) a solution to the differential equation?

  \blanklines{25}
\end{example}
\clearpage

\begin{example}
  Consider the differential equation \( u'(t) + p(t) u(t) = 0\) where \(p(t)\) is a continuous function. Define
  \[
    u(t) = C e^{-\int p(t) \;dt},
  \]
  where \(C\) is a constant. Is \(u(t)\) a solution to the given differential equation?
\end{example}


\begin{example}
  Consider the differential equation \( u'(t) + p(t) u(t) = 1 \) where \(p(t)\) is a continuous function. Is
  \[
    u(t) = \frac{1}{v(t)} \left( \int v(t) \;dt + C \right), \quad\text{where}\quad v(t) = e^{\int p(t)\;dt}
  \]
  a solution to the given differential equation?
  \clearpage
\end{example}

Example~\ref{ex:general-solution-of-linear-first-order-diff-eq} develops our skills to work with abstract statements. 

\begin{example} \label{ex:general-solution-of-linear-first-order-diff-eq}
  Let \(p(t), q(t)\) be continuous functions.  Define another function \(v(t) = e^{\int p(t) \;dt}\). 

  Verify that \(u(t) = \frac{1}{v(t)} \left( \int v(t) q(t) \;dt + C \right)\) is a solution to the differential equation
  \[
    u'(t) + p(t) u(t) = q(t).
  \]

  \clearpage
\end{example}

Our ability to verify a proposed solution allows us to solve differential equations by simply guess and check. Such an exercise gives us a little bit of intuition on how our prior knowledge (differentiation and integration) leads to a general solution some simple differential equations.
\begin{example}
  Assume \(y\) is a function of \(x\). Solve \(xy' + y = x\) by guess and check.

  Hint: The left-hand side is the derivative of \(xy\) using implicit differentiation. Therefore, the differential equation can be rewritten as \(\frac{d}{dx} \left( xy \right) = x\). What happens if we integrate both sides?
\end{example}
\end{document}
