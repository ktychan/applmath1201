%! TeX program = lualatex
\documentclass[../main.tex]{subfiles}
\begin{document} \section{Log-log transformation}

Visualization is often a helpful method to \emph{make sense} of data collected from empirical studies. However, how can we discern qualitative features from data clump together?

\begin{example}
  Consider the following data sets. 

  \begin{luacode}
    function generate_noisy_data(f, x_min, x_max, num_samples, noise_level)
      local data = {}

      -- Generate evenly spaced x values between x_min and x_max
      local step = (x_max - x_min) / (num_samples - 1)

      for i = 0, num_samples - 1 do
        local x = x_min + i * step
        local y = f(x)

        -- Add random noise to y (perturbation)
        local noise = math.random() * 2 * noise_level - noise_level  -- noise between -noise_level and noise_level
        y = y + noise

        table.insert(data, {x = x, y = y})
      end

      return data
    end

    function print_data(data)
      for _, point in ipairs(data) do
        tex.print(string.format("%.2f , %.2f\\\\", point.x, point.y))
      end
    end
  \end{luacode}

  \begin{tabular}{c}
    \(x\) , \(y\) \\\midrule
    \directlua{print_data(generate_noisy_data(function(x) return x + 1 end, 1, 2, 10, 0.1))}
  \end{tabular}


  \begin{tabular}{c}
    \(x\) , \(y\) \\\midrule
    \directlua{print_data(generate_noisy_data(function(x) return 0.5*x*x end, 1, 2, 10, 0.1))}
  \end{tabular}


  \begin{tabular}{c}
    \(x\) , \(y\) \\\midrule
    \directlua{print_data(generate_noisy_data(function(x) return 0.1*x*x*x*x end, 1, 2, 10, 0.1))}
  \end{tabular}
\end{example}

\begin{mdframed}[style=simple]
  \textbf{Idea 2}. Suppose we have a data set with a best-fit model \(y = c x^{\alpha}\). 
  The log-log transformation of such a data set, a plot \(\ln(x)\) against \(\ln(y)\), is a straight line whose slope is \(\alpha\) and \(y\)-intercept is \(\ln(c)\).

  Conversely, in the log-log space, if a straight line \(\ln(y) = y_{0} + \alpha \ln(x)\) is the best fit of a log-log transformed data set, then the model \(y = e^{y_{0}} x^{\alpha}\) fits the original data best.
\end{mdframed}

\begin{example}
  Find the best-fit model for the following data in log-log space.

  \todo{TODO: make a plot}.
\end{example}

\begin{example}
  Sketch the model \(y = 2 \sqrt{x}\) in the log-log space.

  \todo{TODO: make a blank log-log plot}.
\end{example}
\end{document}
