%! TeX program = lualatex
\documentclass[../main.tex]{subfiles}
\begin{document} \section{Log-log transformation}

Visualization is often a helpful method to \emph{make sense} of data collected from empirical studies. If we suspect a phenomenon to follow a power rule model \(y = c x^{\alpha}\), we can use the log-log transformation to determine the required exponent \(\alpha\) and coefficient \(c\).

\begin{definition}[log-log transformation]
  The (natural) log-log transformation of an equation \(y = c x^{\alpha}\) is
  \[
    \ln(y) = \ln(c) + \alpha \ln(x)
  \]
  obtained by applying \(\ln(\cdots)\) to both sides of \(y = c x^{\alpha}\).
\end{definition}

\blanklines{5}


\begin{definition}[log-log space]
  A point \((x,y)\) in the Euclidean space corresponds to \((\ln(x), \ln(y))\) in the log-log space, and a point \((\alpha,\beta)\) in the log-log space corresponds to \((e^{\alpha}, e^{\beta})\) in the Euclidean space.
\end{definition}

\begin{figure}[H] % [h] for here, [ht] for here top, [hb] for here bottom
  \centering
  
  \hfill{}
  \includegraphics{../standalones/build/plot-blank-5x5.pdf}
  \hfill{}
  \includegraphics{../standalones/build/plot-log-log-5x5-plain.pdf}
  \hfill{}

  \caption{Euclidean space vs log-log space}
  \label{fig:figure}
\end{figure}

\begin{example}
  Sketch the points \((20,400), (50, 2500), (100,10000) \) in the above spaces. Use a calculator to approximate \(\ln(\cdots)\) values.

  \blanklines{3}
\end{example}

\faCalculator{} Be aware that most online calculators perform \emph{base \(10\)} log-log transformations instead of natural log-log transformations.  In particular, Desmos' log-log transformation has base 10. 

See \url{https://upload.wikimedia.org/wikipedia/commons/2/29/Solarmap.gif} for a cool application of the base 10 log-log transformation.


\begin{example}
  Sketch the model \(y = 0.0135 (\sqrt{x})^{3}\) in the log-log space.  Extend the grid if necessary.

  \begin{center}
    \includegraphics{../standalones/build/plot-log-log-5x5-plain.pdf}
  \end{center}

  \blanklines{10}
\end{example}

\begin{example}
  Find the best-fit model for the following data in log-log space.

  \todo{TODO: make a plot}.
\end{example}
\end{document}
