%! TeX program = lualatex
\documentclass[../main.tex]{subfiles}
\begin{document} \section{Allometry}
In Example~\ref{ex:toucan}, we used the relation \[\text{weight} = \text{density} \times \text{volume} \] to guide our choice of a model. 

\begin{mdframed}[style=simple]
  If two quantities \(x,y\) are related by a \hlattn{linear} relation \(y = c x\) for some constant \(c > 0\), then we say \hlmain{\(x\) and \(y\) exhibit isometry}.

  If two quantities \(x,y\) are related by a \hlattn{non-linear} relation, i.e., \(y = c x^{\alpha}\) where \(\alpha \ne 1\), then we say \hlmain{\(x\) and \(y\) exhibit allometry}.
\end{mdframed}

\begin{example}
  Consider the length, volume and weight of birds as discussed in Example~\ref{ex:toucan}.

  \begin{enumerate}[wide]
    \item Do length and weight exhibit isometry or allometry? Explain your reasoning.
      \blanklines{10}

    \item Do volume and weight exhibit isometry or allometry? Explain your reasoning.
      \blanklines{10}
  \end{enumerate}
\end{example}
\end{document}
