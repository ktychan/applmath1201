%! TeX program = lualatex
\documentclass[../main.tex]{subfiles}
\begin{document} \section{Functions of vectors}

A function is a real-valued function if the output is a real number. These are our familiar functions.  For example, 
\[
  f(\vec{x}) = e^{x_{1} + x_{2}}
\]
is a real-valued function because its output is a real number. Its input does not matter. 

A function is a vector-valued function if the output is a vector (with dimension \(2\) or more). For example, these are vector-valued functions
\[
  f(\vec{x}) = 
  \begin{bmatrix}
    x_{1}^{2} \\
    x_{2}^{2}
  \end{bmatrix} 
  \qquad\text{and}\qquad
  g(\vec{x}) = 
  \begin{bmatrix}
    0.1 x_{1} x_{2} + x_{1} \\
    0.7 x_{1} + 0.8 x_{1} x_{2} \\
  \end{bmatrix}.
\]

In particular, the notation for vector-valued functions are just a compact way to write a couple equations together. The function \(g(\vec{x})\) can be written as 
\begin{align*}
  y_{1} &= 0.1 x_{1} x_{2} + x_{1} \\
  y_{2} &= 0.7 x_{1} + 0.8 x_{1} x_{2}
\end{align*}
where \(x_{1}, x_{2}\) are inputs and \(y_{1}, y_{2}\) are outputs. In other words, we are just writing \(g(\vec{x}) = \begin{bmatrix} y_{1} \\ y_{2} \end{bmatrix}\).

\faStar{} A vector-valued function \(f(\vec{x})\) is \hlmain{linear} if it satisfies 
\begin{equation}
  f(\vec{0}) = \vec{0} 
  \quad\text{and}\quad
  f(a \vec{x} + b \vec{y}) = a f(\vec{x}) + b f(\vec{y}),
\end{equation}
for all scalers \(a,b\) and all vectors (of the same dimension) \(\vec{x}, \vec{y}\).

Linear vector-valued functions are \sout{particularly} extremely nice.  There is a ton of \sout{mathematical theory} software capable of doing all sort of things with linear vector-valued functions. Therefore, we should identify linear vector-valued functions.

\begin{example}
  Show that \(f(\vec{x}) = \begin{bmatrix} x_{1} + 3x_{2} \\ -x_{1} + x_{2} \end{bmatrix}\) is a linear vector-valued function. 
\end{example}

\begin{example}[Example~6 on page 157 of the textbook]
  Consider a population of juveniles and adults.  At the end of each year, 
  \begin{itemize}
    \item \(20\%\) of the surviving juveniles become adults, 
    \item \(90\%\) of the adults survive, and
    \item the per-adult-capita birth rate is \(15\%\). 
  \end{itemize}

  Encode the age-structured population dynamics as a vector field and predict the number of adults and juveniles after \(2\) years if the initial population consists of \(2\) juveniles and \(4\) adults.

  \blanklines{5}
\end{example}
\end{document}
