%! TeX program = lualatex
\documentclass[../main.tex]{subfiles}
\begin{document} \section{Functions of vectors}

An equation of the form \(A \vec{x} = \vec{b}\) takes vectors as inputs and spits out vectors. We can write that in the form as a function.
\blanklines{10}

Such a function is called a \hlmain{vector-valued function} because their outputs are vectors.  

A function is a real-valued function if the output is a real number. These are our familiar functions.  For example, 
\[
  f(\vec{x}) = e^{x_{1} + x_{2}}
\]
is a real-valued function because its output is a real number. Its input does not matter. 

A function is a vector-valued function if the output is a vector (with dimension \(2\) or more). For example, these are vector-valued functions
\[
  f(\vec{x}) = 
  \begin{bmatrix}
    x_{1}^{2} \\
    x_{2}^{2}
  \end{bmatrix} 
  \qquad\text{and}\qquad
  g(\vec{x}) = 
  \begin{bmatrix}
    0.1 x_{1} x_{2} + x_{1} \\
    0.7 x_{1} + 0.8 x_{1} x_{2} \\
  \end{bmatrix}.
\]

In particular, the notation for vector-valued functions are just a compact way to write a couple equations together. The function \(g(\vec{x})\) can be written as 
\begin{align*}
  y_{1} &= 0.1 x_{1} x_{2} + x_{1} \\
  y_{2} &= 0.7 x_{1} + 0.8 x_{1} x_{2}
\end{align*}
where \(x_{1}, x_{2}\) are inputs and \(y_{1}, y_{2}\) are outputs. In other words, we are just writing \(g(\vec{x}) = \begin{bmatrix} y_{1} \\ y_{2} \end{bmatrix}\).

\faStar{} A vector-valued function \(f(\vec{x})\) is \hlmain{linear} if it satisfies 
\begin{equation}
  f(c \vec{v}) = c f(\vec{v})
  \quad\text{and}\quad
  f(\vec{x} + \vec{y}) = f(\vec{x}) + f(\vec{y}),
\end{equation}
for all scalers \(c\) and all vectors \(\vec{x}, \vec{y}\) having the same dimension.

\end{document}
