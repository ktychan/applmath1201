%! TeX program = lualatex
\documentclass[../main.tex]{subfiles}
\begin{document} \section{Functions of vectors}

We dip out toes in the general idea of vector-value functions. 

A function is a real-valued function if the output is a real number. These are our familiar functions.  For example, 
\[
  f(\vec{x}) = e^{x_{1} + x_{2}}
\]
is a real-valued function because its output is a real number. Its input does not matter. 

A function is a vector-valued function if the output is a vector (with dimension \(2\) or more). For example, 
\[
  f(\vec{x}) = 
  \begin{bmatrix}
    x_{1}^{2} \\
    x_{2}^{2}
  \end{bmatrix}
\]
is a vector-valued function. 

One of the major differences between real-value functions and vector-value functions is that a vector-value function (in which the input and output vectors have the same dimension) can be used to encode multi-dimensional data such as our orca-seal data.

To visualize a vector-valued function, we draw the output vector as a line segment originating from the point of the input vector.

\end{document}
