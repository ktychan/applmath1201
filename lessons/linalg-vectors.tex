%! TeX program = lualatex
\documentclass[../main.tex]{subfiles}
\begin{document} 
\section{Vectors}

We have been looking at models involving one quantity, e.g., the population of a group of unicorns.  We now develop mathematical tools (\emph{linear} vector fields and \emph{linear} systems of differential equations) to model systems involving two or more quantities, e.g., an ecosystem involving a predator and prey.

\begin{example}
  We begin with an ecosystem involving orcas (predators) and seals (preys). Their population data over time are presented as graphs.

\end{example}

\vfill{}

We study vectors and vector-valued functions because their graphs look like the one on the right and can be used to encode states, relations and changes of biological systems.

\begin{definition}[vectors]
  A (state) \hlmain{vector} \(\vec{x}\) is an ordered list of numbers
  \[
    \vec{x} = 
    \begin{bmatrix}
      x_{1} \\ x_{2} \\ \vdots \\ x_{n}
    \end{bmatrix}
    = 
    \begin{bmatrix}
      x_{1} & x_{2} & \cdots & x_{n}
    \end{bmatrix}^{T}.
  \]

  The number of rows in \(\vec{x}\) (zeros included) is called the \hlmain{dimension} of \(\vec{x}\).   
\end{definition}
\clearpage

\section{Operations on vectors}

We study vectors abstractly. Every vector \(\vec{v}\) has a \hlmain{length} defined by 
\begin{equation}
  \| \vec{v} \| = \sqrt{ (v_{1})^{2} + \cdots + (v_{n})^{2} }.
\end{equation}

\includegraphics{../standalones/build/plot-RR2}

The following two operations are often defined together because they have nice geometric meanings. \begin{equation}
  c
  \begin{bmatrix}
    u_{1} \\ u_{2} \\ \vdots \\ u_{n}
  \end{bmatrix}
  =
  \begin{bmatrix}
    c u_{1} \\ c u_{2} \\ \vdots \\ cu_{n}
  \end{bmatrix}
  \quad\text{and}\quad
  \vec{u} + \vec{v} 
  =
  \begin{bmatrix}
    u_{1} \\ u_{2} \\ \vdots \\ u_{n}
  \end{bmatrix}
  +
  \begin{bmatrix}
    v_{1} \\ v_{2} \\ \vdots \\ v_{n}
  \end{bmatrix}
  =
  \begin{bmatrix}
    u_{1} + v_{1} \\ v_{2} + v_{2} \\ \vdots \\ u_{n} + v_{n}
  \end{bmatrix}.
\end{equation}
The constant \(c\) is called a \hlmain{scalar} because it scales a vector's length without rotation.  This effect is expressed in the identity \underline{\hspace{1.5in}\phantom{\huge X}}.
\bigskip

\includegraphics{../standalones/build/plot-RR2}
\quad
\includegraphics{../standalones/build/plot-RR2}
\quad
\includegraphics{../standalones/build/plot-RR2}

\blanklines{10}
\clearpage

In general, we cannot multiply vectors. However, we can define a special product that takes two vectors and produces a single number. 

\begin{definition}[dot product]
  The \hlmain{dot product} of two vectors of the same dimension is defined by 
  \begin{equation}
    \vec{u} \bullet \vec{v} = u_{1} v_{1} + u_{2} v_{2} + \cdots + u_{n} v_{n}.
  \end{equation}

  Moreover, \(\vec{u} \bullet \vec{v} = \|\vec{u}\| \|\vec{v}\| \cos(\text{angle between \(\vec{u}\) and \(\vec{v}\)})\).

  \includegraphics{../standalones/build/plot-RR2}
\end{definition}

\faStar{} Two vectors \(\vec{u}\) and \(\vec{v}\) of the same dimension are equal if \(u_{1} = v_{1}, u_{2} = v_{2}, \dots, u_{n} = v_{n}\).  

\begin{example}
  Find the vector \(\vec{x}\) in the equation \(\begin{bmatrix} 1 \\ 3 \end{bmatrix} + 2 \vec{x} = \begin{bmatrix} -3 \\ 5 \end{bmatrix}\) and calculate \(\| -5 \vec{x} \|\).

  \blanklines{10}
\end{example}

\begin{example}
  Let \(\vec{x} = [ x_{1} \; x_{2} \; x_{3} ]^{T}\).  Write \(-3x_{1} + x_{2} - 3x_{3}\) as a dot product. 

  \blanklines{5}
\end{example}

\begin{example}
  Let \(\vec{x} = [ x_{1} \; x_{2} \; x_{3} ]^{T}\).  Write \(- x_{1} + \tfrac{1}{2}x_{3}\) as a dot product. 
  
  \blanklines{5}
\end{example}

\begin{example}
  Consider the equation \(\vec{u} \bullet \vec{v} = 4\). What is \(\|\vec{u}\|\) if \(\vec{v} = \left(\tfrac{1}{\sqrt{2}}, \tfrac{1}{\sqrt{2}}\right)\) and the angle between them is \(\pi/6\)?

  \blanklines{10}
\end{example}

For completeness, here is a frequently used terminology in the study of vectors.

A linear combination of (finitely many) vectors \(\vec{u}, \vec{v}, \vec{w}, \ldots\) is 
\[
  a \vec{u} + b \vec{v} + c \vec{w} + \cdots, \text{ where \(a,b,c, \ldots\) are scalers}.
\]

Also for completeness, we can take two vectors \(\vec{u}, \vec{v}\) and define a third vector called their cross product 
\[
  \vec{u} \times \vec{v} = 
  \begin{bmatrix}
    u_{2} v_{3} - u_{3} v_{2} \\
    - u_{1} v_{3} + u_{3} v_{1} \\
    u_{1} v_{2} - u_{2} v_{1} \\
  \end{bmatrix}.
\]

For more details of the cross product, see pages 149 to 151 of the textbook. 

\end{document}
