%! TeX program = lualatex
\documentclass[../main.tex]{subfiles}
\begin{document} \section{Linearity}

We learn a bit more theory of vectors and matricies to prepare for solving systems of recurrence equations and systems of differential equations.

An equation of the form \(A \vec{x} = \vec{b}\) takes vectors as inputs (left-hand side) and spits out vectors (right-hand side). This interpretation says matrix-vector multiplication is a function \(F(\vec{x}) = A \vec{x}\). Such functions are called vector-valued functions because their outputs are vectors.

\blanklines{10}

We learn to work with a particularly simple class of vector-valued functions called linear functions.

\begin{definition}[linearity] \label{def:linearity}
  A function \(f(\vec{x})\) is called \hlmain{linear} if \(F(\vec{x}) = A\vec{x}\) for some matrix \(A\). If a function is not a matrix-vector multiplication, then it is not linear. All linear functions satisfy the property
  \begin{equation} \label{eq:linearity}
    A(a \vec{u} + b \vec{v}) = a A\vec{u} + b A\vec{v} \quad\text{where \(a,b\) can be any constants.}
  \end{equation}
\end{definition}
\blanklines{5}

Equation~\eqref{eq:linearity} seems counter-intuitive because why perform two matrix-vector multiplications (right-hand side) when you can just do it once (left-hand side).  Matricies have special vectors, called \emph{eigenvectors}, for which matrix-vector multiplications are \emph{extremely} simple.  In \thecoursesubject{}~\thecoursenumb{}, we want to rewrite a vector in terms of eigenvectors and apply Equation~\ref{eq:linearity} to simplify our computation for \(A^{n} \vec{x}\). We will learn eigenvectors next week. For now, we learn to rewrite vectors in terms of eigenvectors of matricies.

\begin{method}[express a vector in terms of other vectors] \label{method:linear-combination}
  To find unknown constants \(a,b\) so that \(\vec{x} = a \vec{u} + b \vec{v}\) where \(\vec{x}, \vec{u}, \vec{v}\) are given, rewrite the equality as (usual) linear equations involving unknown \(a,b\), then use back substitution.
\end{method}

\faStar{} In \thecoursesubject{}~\thecoursenumb{}, Method~\ref{method:linear-combination} is often used in conjunction with Equation~\ref{eq:linearity}. However, they are not married to each other in general. 

{\footnotesize \faExclamationTriangle{} In \thecoursesubject{}~\thecoursenumb{}, such problems will always have solutions, because we only deal with ``special enough'' vectors \(\vec{u}, \vec{v}\). However, such equations need not have a solution if \(\vec{u}, \vec{v}\) are not ``special enough.'' For those interested, the linear algebraic concept for ``special enough'' is called \emph{linear independence}.}
\clearpage

Consider a matrix \(A = \begin{bmatrix} 0 & 2 \\ 1 & 1 \end{bmatrix}\) and its two eigenvectors \(\vec{u} = \begin{bmatrix} 1 \\ 2 \end{bmatrix}\) and \(\vec{v} = \begin{bmatrix} 1 \\ 0 \end{bmatrix}\).  Our goal is to find \hlmain{a simple answer} for arbitrary powers \(A^{n} \vec{x}\) where \(\vec{x} = \begin{bmatrix} -11 \\ 10 \end{bmatrix}\).

\begin{example} \label{ex:linear-combination}
  Find unknown constants \(a,b\) so that \(\vec{x} = a \vec{u} + b \vec{v}\). Hint: Use Method~\ref{method:linear-combination}.

  \blanklines{20}
\end{example}

\begin{example}
  Calculate \(A^{n} \vec{u}\) and \(A^{n} \vec{v}\) for arbitrary positive integers \(n\). 

  \blanklines{15}
\end{example}

\begin{example}
  Calculate \(A^{n} \vec{x}\) for arbitrary positive integers \(n\). Hint: Use Equation~\eqref{eq:linearity}.

  \blanklines{5}
\end{example}
\clearpage

\end{document}
