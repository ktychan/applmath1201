%! TeX program = lualatex
\documentclass[../main.tex]{subfiles}
\begin{document} \section{Mathematical techniques for recursive models}

Now that we know how to turn words into recursive models, we can start to think about techniques to investigate recursion equations. 

At the end of this section, we will be able to do the following.
\begin{enumerate}
  \item Given a recursion, computer a few terms to get an sense of its behaviour.
  \item Verify proposed solutions of recursions.
  \item Understand the effect of initial conditions and find equilibrium solutions.
  \item Solve simple recursions.
\end{enumerate}

\faLightbulb{} Although Example 2.5 is not a biological model, it is intuitive and sophisticated enough to demonstrate all of this section's learning goals.

To calculate a few terms of a given recurrence, we simply start with time \(t=1\) and calculate \(u_{1}\) using the recurrence equation. Once \(u_{1}\) is calculated, we apply \(t = 2\) to the recurrence equation to calculate \(u_{2}\) and so on.
\begin{example} \label{ex:recurrence-bruteforce}
  Calculate the balance at the end of years 1, 2, and 3 using the recurrence in Example~\ref{ex:recurrsion-investment-model}.
  \blanklines{15}
\end{example}

\begin{example}
  Calculate the balance at the end of years 3 using the recurrence in Example~\ref{ex:recurrsion-investment-model} but change the initial deposit to \(5000\) dollars.
  \blanklines{10}
\end{example}

\clearpage

Theoretically, we can always use the recursion equation to calculate \(u_{t}\) for any arbitrary \(t\), say \(t = 10^{30}\), of a recursion. However, this approach is time consuming (even for a computer if a recursion is sufficiently complicated). Moreover, this approach does not give us any insight into the behaviour of the model. We can't really tell if a model follows a power rule, is exponential or something else completely. 

In some cases, we can use mathematical techniques to ``get rid'' of the recursion equation and directly expression \(u\) as a function of \(t\). This is called \hlmain{solving a recursion}.

First, let's understand what it means for a function to be a solution of a recursion. 

\begin{definition}[solutions of recursions]
  Suppose \(u_{t}\) satisfies a recursion with some initial condition \(u_{0}\). A function \(u = f(t)\) is a solution of the recursion if 
  \begin{enumerate}
    \item \(f(t)\) satisfies the defining recursion equation for \(u_{t}\), \emph{and}
    \item \(u_{0} = f(0)\).
  \end{enumerate}

  Note that \(f(t)\) cannot refer to any \(u_{t}\) and must be a well-defined function of \(t\).
\end{definition}

\begin{example}
  Verify that \(f(t) = 1000 (1.02)^{t}\) is \emph{not} a solution of the recursion in Example~\ref{ex:recurrsion-investment-model}.
  \blanklines{30}
\end{example}
\clearpage

To denote a solution, we can skip the function notation and simply drop subscripts.
\begin{example}
  Verify that \(b = -1500 (1.02)^{t} + 2500\) is a solution of the recurrence in Example~\ref{ex:recurrsion-investment-model}.
  \blanklines{50}
\end{example}
\clearpage

We now turn to the third learning goal of this section: Initial conditions and equilibrium solutions. 

Recurrences are sensitive to their initial condition. Consider once again Example~\ref{ex:recurrsion-investment-model}. Recall our savings account pays \(2\%\) annual interest rate, and we withdrawal \(50\) dollars after each interest payment.  Initial deposits of \(1000\) vs \(2500\) should lead to different balances after \(10\) years.  This is true for all recurrences: Different initial conditions lead to different solutions.

However, the initial deposit \(2500\) is special. It leads to no change from year to year.

\begin{definition}[equilibrium solution]
  An equilibrium solution of a recurrence equation (not the whole recurrence) is a constant function.  We denote equilibrium solutions by making the variable wear a hat and throwing away subscripts, e.g., \(\hat{u}\).
\end{definition}

\begin{example}
  Verify that \(\hat{b} = 2500\) is an equilibrium solution of \(b_{t} = 1.02 b_{t-1} - 50\).  Moreover, explain in plain English why this makes sense in the context of Example~\ref{ex:recurrsion-investment-model}.
  % results in no change in states
  \blanklines{18}
\end{example}

\faStar{} Equilibrium solutions are very easy to find because they are constants. Simply replace all \(u_{t}\) by \(\hat{u}\) and solve for \(\hat{u}\) using basic algebra. It is possible to find more than one equilibrium solutions.

\begin{example}
  Find \emph{all} equilibrium solutions of the recurrence equation \(u_{t} = \sqrt{3 u_{t-1} - 2}\).
  \blanklines{15}
\end{example}
\clearpage

\begin{example}
  Find equilibrium solution(s) of the recurrence equation \(p_{t} = \frac{1}{2} p_{t-1} - \frac{1}{2}\).
  \blanklines{15}
\end{example}

Our last learning goal is to solve recurrence.

Guess and check is the simplest technique to solve recursions. However, it is not very useful in general and only works for very simple recursions.
\begin{example}[important] \label{ex:recurrence-exp}
  Let \(a,\lambda\) be constants. Assume \(\lambda > 0\). The recurrence 
  \[ 
    u_{0} = a \quad\text{and}\quad u_{t} = \lambda u_{t-1}, \text{ for integers } t \ge 1 
  \] 
  forms the foundation of solving recurrence in this course. 
  
  Solve this recurrence by guess and check.
  \blanklines{20}
\end{example}

\clearpage

Change-of-variable is the general techniques we will use in this course. 

\begin{method}
  Given a recurrence of the form \(v_{t} = A v_{t-1} + B\), we wish to make a change of variable \(u_{t} = v_{t} + c\) so that \(u_{t}\) is the recurrence in Example~\ref{ex:recurrence-exp}. Solve the new recurrence and recover the original solution.
\end{method}

\begin{example}
  Solve the recurrence \(b_{t} = 1.02 b_{t-1} - 50\) with initial condition \(b_{0} = 1000\).

  \begin{enumerate}[wide, label={Step~(\arabic*).}]
    \item Perform the change of variable. The two methods below lead to the same solution.
      \begin{enumerate}[label={(\alph*)}]
        \item Calculate the equilibrium solution \(\hat{u}\), and the change of variable is \(b_{t} = u_{t} - \hat{u}\).
        \item Blindly perform the change of variable \(b_{t} = u_{t} - c\), then set \(c\) to be the number so that the constant term is \(0\).
      \end{enumerate}
      \blanklines{20}

    \item Calculate \(u_{0}\) and solve for the new recurrence in \(u_{t}\). 
      \blanklines{10}

    \item Recover the solution of the original recurrence. Verify that we didn't make a mistake.
      \blanklines{5}
  \end{enumerate}
\end{example}
\clearpage

\begin{example}
  Solve the recurrence \(p_{0} = 1\) and \(p_{t} = \frac{1}{2} p_{t-1} - \frac{1}{2}\) for integers \(t \ge 1\).

  \blanklines{50}

  {\footnotesize The solution is \(p = \frac{2}{3} \left( -\frac{1}{2} \right)^{t} + \frac{1}{3}\) for all integers \(t \ge 0\).}
\end{example}

\clearpage

Solutions of recurrence allows us to answer interesting questions. 
\begin{example}
  Consider the recurrence in Example~\ref{ex:recurrsion-investment-model} 
  \[
    b_{0} = 1000 \quad\text{and}\quad b_{t} = 1.02 b_{t-1} - 50, \text{ for integers } t \ge 1. 
  \] 
  Does the balance ever reach \(0\)? If so, roughly how many years since the initial investment should we expect the balance to be \(0\)? 
  
  Don't simply list all \(b_{t}\). Use the solution of the recurrence and basic algebra to solve this problem.
\end{example}
\end{document}
