%! TeX program = lualatex
\documentclass[../main.tex]{subfiles}
\begin{document} \section{Dimensional homogeneity}

Mathematical models of physical phenomena are often presented as mathematical equations. We learn a quick sanity-check to make sure our model is comparing apples to apples.

A model should make sense physically. We do so by checking that the units on both sides of the equation match.

\begin{mdframed}[style=simple-compact]
  A model, expressed as a mathematical equation, is \hlmain{dimensionally homogeneous} if the units associated with each of its \emph{terms} match.
\end{mdframed}

A ``term'' is a summand in the equation.

\faPencil*{} ``Dimensionally homogeneous'' is also known as ``dimensionally consistent'' (often in physics).

\begin{example}
  It is often said that ``time is money.'' Expressed as a mathematical model, we have 
  \[
    t = M
  \]
  where \(t\) is time (minutes) and \(M\) is money (dollars).

  Is this model dimensionally homogeneous?
  \blanklines{8}
\end{example}

A dimensionally homogeneous model provides useful information. 
\begin{example}
  Einstein's model \(E = mc^{2}\) is dimensionally homogeneous. The symbol \(E\) represents energy whose unit is \(\text{kg} \cdot \text{m}^{-2} \cdot \text{s}^{-2}\). The symbol \(m\) represents mass whose unit is kg. 

  What is the unit of \(c\)?
  \blanklines{10}
\end{example}

\begin{example}
  Find the missing unit so that the model is dimensionally homogeneous.
\end{example}
\end{document}
