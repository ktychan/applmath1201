%! TeX program = lualatex
\documentclass[../main.tex]{subfiles}
\begin{document} \section{Introduction to systems of differential equations}

A system of differential equations consists of multiple equations of differentiable functions and their derivatives.  Unknown functions can depend on each other and their derivatives to express relations among quantities.

\begin{example}[Lokta-Volterra Equations]
  Next week, we will run into the Lokta-Volterra equations
  \begin{align*}
    x_{1}'(t) &= \alpha x_{1}'(t) - \beta x_{1}'(t) x_{2}'(t) \\
    x_{2}'(t) &= \gamma x_{1}'(t) x_{2}'(t) - \delta x_{2}'(t).
  \end{align*}

  The dependent variables (unknown functions) in the Lokta-Volterra equations are \underline{\hspace{1in}}, and the independent variable is \underline{\hspace{1cm}}.
\end{example}

We use compact notations \(\frac{d \vec{x}}{dt}\) or \(\vec{x}'(t)\) to express the derivatives of the components of \(\vec{x}(t)\).   By definition, we have
\[
  \frac{d \vec{x}}{dt} = \begin{bmatrix} x_{1}'(t) \\ \vdots \\ x_{n}'(t)\end{bmatrix}.
\]


\begin{example}
  Suppose \(x_{1}(t) = 3 e^{5t} -2 e^{t/2}\) and \(x_{2}(t) = 2 e^{5t} + \frac{1}{2}e^{t/2}\). Find \(\frac{d \vec{x}}{dt}\) and express the result as a linear combination of vectors.

  \blanklines{18}
\end{example}

\faStar{} Classification of systems of DEs are similar to that of differential equations. However, notice the use of ``only'' and ``some'' in the definitions below.
\begin{itemize}
  \item A first-order system contains \hlmain{only} first-order differential equations.
  \item A linear system contains  \hlmain{only} linear differential equations.
  \item A non-linear system contains \hlsupp{some} non-linear differential equations.
  \item A first-order, linear system contains \hlmain{only} first-order, linear differential equations.
\end{itemize}

For example, the Lokta-Volterra equations are \underline{\hspace{3in}}.
\clearpage

Linear, first-order systems of DEs can be expressed as a matrix equation \(\frac{d \vec{x}}{dt} = A \vec{x}(t) + \vec{b}\) where \(\vec{b}\) only has constants.

\begin{example}
  Express the following linear, first-order system of DEs as a matrix equation.
  \[
     \begin{array}{rcrcr}
      x_{1}'(t) &=&  2 x_{1}(t) \\
      x_{2}'(t) &=& -6 x_{1}(t) &-& x_{2}(t) \\
     \end{array}
     \quad\text{with initial condition}\quad
     \begin{array}{rcr}
      x_{1}'(0) &=& 4 \\
      x_{2}'(0) &=& 2
     \end{array}
  \]

  \blanklines{10}
\end{example}

\begin{example}
  Express the following linear, first-order system of DEs as a matrix equation.
  \[
     \begin{array}{rcl}
       x_{1}'(t) &=& 0.3 - 0.1 x_{2}(t) \\
       x_{2}'(t) &=&   5 x_{1}(t) + x_{2}(t) + 0.2 \\
     \end{array}
     \quad\text{with initial condition}\quad
     \begin{array}{rcr}
      x_{1}'(0) &=& 4 \\
      x_{2}'(0) &=& 2
     \end{array}
  \]

  \blanklines{10}
\end{example}

An equilibrium solution for a system of DE is a vector of constant functions, denoted as \(\vec{x}^{*} = \begin{bmatrix} x_{1}^{*} \\ x_{2}^{*} \end{bmatrix}\).
\begin{example}
  Find equilibrium solutions for 
\end{example}

  \vec{x}(t) = \hlsupp{\alpha} \cdot \underbrace{e^{\mu t} \vec{u}}_{ \parbox[c]{1.6in}{\centering fundamental solution associated to \(\vec{u}\)}  } + \hlsupp{\beta} \cdot \underbrace{e^{\lambda t} \vec{v}}_{ \parbox[c]{1.6in}{\centering fundamental solution associated to \(\vec{v}\)} }.
\end{document}
