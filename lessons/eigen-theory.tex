%! TeX program = lualatex
\documentclass[../main.tex]{subfiles}
\begin{document} \section{Finding eigenvalues and eigenvectors}

Recall that linear vector fields are functions of the form \(f(\vec{x}) = A\vec{x}\) where \(A\) is a matrix of an appropriate dimension.  

In particular, we are interested in understanding ``how do (state) vectors move inside a vector field?''

TODO add two plots.

\begin{definition}[eigenvalues and eigenvectors]
  Suppose \(A\) is a square matrix \(A\). An eigenvalue-eigenvector pair is a solution to the equation
  \[
    A \vec{v} = \lambda \vec{v},
  \]
  where \(\lambda\) is a scalar and \(\vec{v}\) is a \hlwarn{non-zero} vector. 

  Moreover, eigenvalues of \(A\) are roots of the equation (whose unknown is \(\lambda\))
  \[
    \det( A - \lambda I ) = 0,
  \]
  where \(I\) is the identity matrix having the same dimension as \(A\).
\end{definition}

\begin{example}
  Find eigenvalues of the matrix 
  \(
    A = \begin{bmatrix}
      1 & -1 \\
      2 & 3
    \end{bmatrix}
  \).
\end{example}

\begin{example}
  Find eigenvalues of the matrix 
  \(
    A = \begin{bmatrix}
      1 & -1 \\
      1 & 1
    \end{bmatrix}
  \).
\end{example}
\end{document}
