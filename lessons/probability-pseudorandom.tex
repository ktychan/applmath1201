%! TeX program = lualatex
\documentclass[../main.tex]{subfiles}
\begin{document} \section{Pseudorandom variable}

In this section, we discuss how to randomly sample data following a certain random variable and introduce pseudorandom variables, which will be used for simulation in the next section.

In breaking with tradition, we learn pseudorandom variables through a mix of mathematics and scientific computing. 

\begin{definition}[integer division]
  Let \(m\) be a positive integer. 

  If \(x\) is an integer, then we can \hlmain{divide \(x\) by \(m\)} and calculate
  \[
    \underbrace{r = x \mod{m}}_{\text{math notation}} 
    \quad\text{and}\quad
    \underbrace{\texttt{r = x \% m}}_{\text{python code}},
  \] where \(r\) is the \emph{smaller non-negative} remainder of \(x\) when divided by \(m\).  \hlsupp{The result must satisfy \(0 \le r < m\).}

  Moreover, integers \(a,b\) are called \hlmain{equivalent modulo \(m\)} if \((a - b) \mod{m}\) is \(0\)
\end{definition}
\faStar{} You should know how to calculate \(x \mod{m}\) by hand and using a computer.

\begin{example}
  Calculate \(8 \mod 3\) by hand and verify your computation using python.
  \blanklines{5}
\end{example}

% \begin{definition}[{the uniform distribution on \([0,1]\)}]
%   The continuous random variable with PDF \(f(x) = 1\) on \([0,1]\) and \(f(x) = 0\) everywhere else is called the \hlmain{uniform distribution on \([0,1]\), denoted by \(U(0,1)\)}. 
% \end{definition}
%
% \begin{example}
%   Calculate \(E(X)\), \(Var(X)\) and \(\sigma\) for \(X = U(0,1)\). Sketch its PDF and CDF.
%   \blanklines{15}
% \end{example}
% \clearpage
%
% The uniform distribution is the simplest continuous random variable since its PDF is just a constant function on \([0,1]\). To generate data that follow \(U(0,1)\) means to randomly generate numbers on the interval \([0,1]\) \emph{with equal probability}. 

We can use integer division to generate pseudorandom numbers. 
\begin{definition}[pseudorandom number generator] \label{def:pseudorandom-number-generator}
  A pseudorandom number generator parameterized by \(a,b,c,m\) is the sequence of numbers defined by the recurrence
  \begin{equation} \label{eq:pseudorandom-number-generator}
    \quad u_{0} = c \qquad\text{and}\qquad u_{t} = (a u_{t-1} + b) \mod m \quad\text{for every integer \(t \ge 1\)}.
  \end{equation}
\end{definition}

Let's cozy up to Equation~\ref{eq:pseudorandom-number-generator}.
\begin{example}
  Set \(a = 5, b = 1, c = 1, m = 8\) in the pseudorandom generator. Write down the corresponding recurrence and calculate the first \(3\) numbers. 

  \blanklines{10}
\end{example}
\clearpage

Definition~\ref{def:pseudorandom-number-generator} generates fake random numbers because they follow a pattern (hence fake) but is hard to predict what comes next in the sequence (hence random) without knowing \(a,b,c,m\).

In python, pseudorandom numbers are generated using a for-loop.

\begin{listing}[h!]
  \begin{pythoncode}
# fill in the right-hand sides with actual positive integers
# the choices below come from the textbook page 113.
a = 1664525 
b = 1013904223
m = 2 ** 32
c = 0            # your choice: c can be integer between 0 and m-1 (inclusive).
n = 1000         # your choice: how many numbers you want from this generator.

# carry out the computation
u = [0] * n      # create a list with n number of 0's.
u[0] = c         # sets $u_{0}$ to c
for t in range(1, n): 
    u[t] = ( a * u[t-1] + b ) % m    # Equation $\ref{eq:pseudorandom-number-generator}$
# END of the for-loop
  \end{pythoncode}

  \caption{Python code for pseudorandom number generator}
  \label{lst:pseudorandom-number-generator}
\end{listing}

\faExclamationTriangle{} The colon at the end of \pythoninline{for t in range(1, n):} signals the beginning of a for-loop. Every line inside the for-loop needs the \emph{same} number of leading spaces (must be at least \(2\) spaces, typically \(4\)).

\faStar{} We use pseudorandom number generator to randomly sample data following \hlmain{\(U(\alpha,\beta)\) the uniform distribution on \([\alpha, \beta]\)}. \(U(\alpha,\beta)\) is a continuous random variable whose PDF is a constant function on an interval \([\alpha, \beta]\) assuming \(\alpha < \beta\).

\begin{method} \label{method:pseudorandom-U(a,b)}
  A pseudorandom number generator spits out integers \(u_{0}, \ldots, u_{n-1}\) whose values fall between \(0\) and \(m-1\) (inclusive), regardless of the choices for other parameters. Therefore, we can convert generated numbers to random samples \(x_{0}, \ldots, x_{n-1}\) that follow \(U(\alpha,\beta)\) using the formula
  \[
    x_{i} = \alpha + (\beta-\alpha) \frac{u_{i}}{m - 1} \text{ for every integer } i = 0, \ldots, n-1.
  \]
\end{method}

Method~\ref{method:pseudorandom-U(a,b)} is just a sequence of transformations of the interval \([0, m-1]\) (Calc~1000 week 1 material).

\blanklines{10}
\clearpage

\begin{example}
  Suppose a pseudorandom number generator with parameter \(m = 2^{12}\) generated \(100\) numbers \(u_{0}, \ldots, u_{99}\).  
  \begin{enumerate}
    \item Write down the formula that converts the generated numbers to random samples \(x_{0}, \ldots, x_{99}\) that follow \(U(-1, 1)\). 
    \item Modify Listing~\ref{lst:pseudorandom-number-generator} to generate \(x_{0}, \ldots, x_{99}\).
  \end{enumerate}

  \blanklines{5}
\end{example}

\medskip
Here is an extra practice on \(U(\alpha,\beta)\). It is not related to pseudorandom variables.
\begin{example}
  Find the PDF, CDF, expectation, variance and standard deviation for \(U(\alpha, \beta)\) assuming \(\alpha < \beta\).
  \blanklines{30}
\end{example}
\vfill{}
Hint: Remember Example~\ref{ex:distribution-with-unknown-bounds}?
\end{document}
