%! TeX program = lualatex
\documentclass[../main.tex]{subfiles}
\begin{document} \section{The Bernoulli substitution}

A substitution, also known as a change of variable, can often \hlmain{transform} an unfamiliar problem into a familiar one to which existing tools apply. 

\begin{method}[Bernoulli substitution] \label{method:bernoulli-sub}
  The \hlmain{Bernoulli substitution} \(u = y^{1-n}\) applies to \hlsupp{non-linear}, first-order differential equation of the form
  \begin{equation} \label{eq:bernoulli}
    y'(t) + P(t) y(t) = Q(t) y(t)^{n}.
  \end{equation}
\end{method}

\begin{example}
  When should we use the Bernoulli substitution?

  Study the \emph{forms} of the following differential equations and determine the applicability of the Bernoulli substitution in each case. 
  
  \begin{enumerate}[label=(\alph*)]
    \item \(v'(t) + 3v(t) = 1\).
    \item \(v'(t) + t v(t) = v^{2}(t)\).
    \item \(v'(t) + \sqrt{v(t)} = 1\).
    \item \(v(t) v'(t) + v(t) = e^{t}\).
  \end{enumerate}
  \blanklines{15}
\end{example}

\faStar{} The substitution \(u = y^{1-n}\) requires us to substitute \(y'(t)\) in terms of \(u'(t)\). Use \emph{implicit differentiation} to get a relation between \(y'\) and \(u'\).
\blanklines{10}
\clearpage

Bernoulli substitution \emph{transforms} a non-linear, first-order differential equation to a linear one. It works, \emph{in spirit}, like Method~\ref{method:recurrence-sub} in the sense that we solve a simpler problem first and recover the solution later.
\begin{example} 
  Solve the initial-value problem \(y'(t) - \frac{2 y(t)}{t}  = -\big( ty(t) \big)^{2}\) with \(y(1) = 1/2\).
  \blanklines{3}

  \begin{enumerate}[wide, label=Step~(\arabic*).]
    \item Identify the substitution \underline{\hspace{2in}}.
    \item Multiply through by \underline{\hspace{1in}} and perform the substitution to get a new differential equation.
      \blanklines{15}
    \item Find the new initial condition \underline{\hspace{3in}}.
    \item Solve the new initial-value problem \underline{\hspace{3in}}.
      \blanklines{5}
    \item Recover the solution of the original initial-value problem by putting \(y\) back.
      \blanklines{10}
  \end{enumerate}
\end{example}
\clearpage

\begin{example}
  Find the general solution for the logistic equation
  \[
    N'(t) = rN(t) \left( 1 - \frac{N(t)}{K} \right), \text{ where \(r,K\) are constants}.
  \]
  \blanklines{50}
\end{example}


\begin{example}
  Solve the initial-value problem \(y'(t) + \frac{1}{1 - t} y(t)  = \sqrt[5]{y(t)}\) with \(y(2) = 0\).
  \blanklines{50}
\end{example}
\clearpage

Whenever the substitution is \(u = y^{1-\text{(odd number)}}\), we need to remember to recover two distinct solutions at the very last step because \(u = y^{\text{(even power)}}\) has two solutions for \(y\), one positive and one negative.
\begin{example}
  Solve the initial-value problem \(y'(t) + y(t) = y(t)^{3}\) given \(y(0) = 1\).
  \blanklines{50}
\end{example}
\end{document}
