%! TeX program = lualatex
\documentclass[../main.tex]{subfiles}
\begin{document} \section{Long-term behaviour of solutions of autonomous equations}

Recall (from the very first lecture of the semester) that models are \emph{imperfect} representations of natural phenomena. Exact solutions to differential equations may or may not be what we \emph{need}. Sometimes, it is good enough to understand the long-term behaviour of solutions.

We introduce the main idea of this section (see \faStar{}) by studying how logistic equations' equilibria interact with their solutions (plural!). Observe how we build a new idea from existing knowledge. Example~\ref{ex:diff-eq-logistic-fact-check} extends the qualitative analysis of differential equations on pages~\pageref{method:diff-eq-model}~and~\pageref{ex:diff-eq-logistic-fact-check}.

\begin{example}[equilibria of logistic equations] \label{eq:diff-eq-logistic-equation-equilibria}
  Consider \(N'(t) = rN(t) \left( 1 - \frac{N(t)}{K} \right)\).

  Notice \hlsupp{\(t\) does not appear in the right-hand side} of the logistic equation at all. By the terminology introduced at the bottom of page~\pageref{def:autonomous}, we see that the logistic equations are examples of autonomous equations. We can create a graph by using \(N\) as an input to produce a value for \(N'\).

  \begin{figure}[H] % [h] for here, [ht] for here top, [hb] for here bottom
    \centering
    \includegraphics{../standalones/build/plot-logistic-solutions-phase-lines}
    \caption{The phase-line plot of \(N' = rN \left( 1 - \frac{N}{K} \right)\).}
    \label{fig:logistic-solutions-phase-lines}
  \end{figure}

  \faStar{} The useful but not obvious observation is that we can say more than just increasing/no change/decreasing for solutions. Figure~\ref{fig:logistic-solutions-phase-lines} \hlmain{predicts} that if \underline{\hspace{2in}}
  \begin{itemize}
    \item \underline{\hspace{2in}}, then the solution \underline{\hspace{3in}}.
    \item \underline{\hspace{2in}}, then the solution \underline{\hspace{3in}}.
  \end{itemize}

  Does the above prediction match reality? Here are several solutions for the logistic equation.
  \begin{figure}[H] % [h] for here, [ht] for here top, [hb] for here bottom
    \centering
    \includegraphics{../standalones/build/plot-logistic-solutions}
    \label{fig:logistic-solutions}
  \end{figure}

\end{example}

The punchline of Example~\ref{eq:diff-eq-logistic-equation-equilibria} is that predictions similar to that of \faStar{} can be made reliably \hlmain{for all autonomous equations}.  Plots like Figure~\ref{fig:logistic-solutions-phase-lines} are called phase-line plots. 

\begin{definition}[phase-line plot]
  Suppose we have an \emph{autonomous} differential equation
  \[
    u' = g(u),
  \]
  where \(g\) is an expression involving the dependent variable \(u\) only and the independent variable of the differential equation does not appear in \(g\) at all.

  The \hlmain{phase-line plot} of the autonomous differential equation is the graph of \(g\).
\end{definition}

\faExclamationTriangle{} Phase-line plots are ONLY defined for autonomous equations.

We learn to identify equilibria of autonomous differential equations from a phase-line plot. Review page~\ref{def:diff-eq-equilibrium} for the technique to find equilibria algebraically.

\begin{example}
  Find equilibria of \(u'(t) = u(t)(u(t)-1)(2 - u(t))^{2}(1 - u(t)/3)\) algebraically and graphically.
  
  \includegraphics{../standalones/build/plot-phase-lines-example1}
\end{example}

\begin{example}
  Find equilibria of \(y'(x) = (y(x) + 3)^{200} (3y(x) - 2) (1/3 - 5y(x))\) algebraically. Use a graphing software to find them graphically. Make sure both methods lead to the same equilibria.
  \blanklines{10}
\end{example}

\faPencil*{} What is the takeaway of equilibria of autonomous differential equations?
\blanklines{5}

We can now describe the idea in \faStar{} in full generality.

\begin{method}[stability analysis of equilibria of autonomous differential equations] 
  Suppose \(\bar{u}\) is an equilibrium of an autonomous differential equation \(u' = g(u)\). 

  \begin{itemize}
    \item If the phase-line plot is \hlmain{above} the horizontal axis (the \(u\)-axis) to the left of \(\bar{u}\) and \hlsupp{below} the horizontal axis to the right of \(\bar{u}\), then \(\bar{u}\) is called a \hlmain{locally asymptotically stable} equilibrium.

      In such a case, if \(u(\text{\color{attn} any constant})\) is near a \hlmain{locally asymptotically stable} equilibrium \(\bar{u}\), then the long-term behaviour of \(u(t)\) is \(\lim_{t \to \infty} u(t) = \bar{u}\).

      \vspace{1cm}

    \item If the phase-line plot is \hlsupp{below} the horizontal axis (the \(u\)-axis) to the left of \(\bar{u}\) and \hlmain{above} the horizontal axis to the right of \(\bar{u}\), then \(\bar{u}\) is called a \hlmain{locally asymptotically unstable} equilibrium.

      In such a case, if \(u(\text{\color{attn} any constant})\) is near a \hlmain{locally asymptotically unstable} equilibrium \(\bar{u}\), then the long-term behaviour of \(u(t)\) is \(\lim_{t \to \infty} u(t) \ne \bar{u}\).

      \vspace{1cm}

    \item Lastly, an equilibrium can be neither stable nor unstable. Such an equilibrium attracts some nearby solutions but repels some other nearby solutions.
  \end{itemize}
\end{method}
\faExclamationTriangle{} In a stability analysis, we always consider the limit of \(u(t)\) as \(t \to \infty\).

We can summarize stability analysis in a picture. 
\begin{center}
  \includegraphics{../standalones/build/plot-phase-lines-stability-wide}
\end{center}
\blanklines{5}


The stability of an equilibrium \(\bar{u}\) of an autonomous differential equation \(u' = g(u)\) can be described in terms of derivatives of \(g'\) whenever \(g\) is differentiable.
\begin{itemize}
  \item If \(g'(\bar{u}) < 0\), then \(\bar{u}\) is a \underline{\hspace{4in}}.
  \item If \(g'(\bar{u}) > 0\), then \(\bar{u}\) is a \underline{\hspace{4in}}.
  \item If \(g'(\bar{u}) = 0\), then \(\bar{u}\) can be \underline{\hspace{4in}}.
\end{itemize}

\clearpage
We can perform phase-line analysis algebraically. \faPencil*{} Remember, \(u(t)\) always moves towards a stable equilibrium and away from an unstable equilibrium as \(t \to \infty\).

\begin{example}
  Consider the differential equation \(u'(t) = u(t)  (3 - u(t)/2) ( 2 - u(t) )(u(t) -5)\). 
  \blanklines{10}
  \begin{enumerate}[wide]
    \item Evaluate \(\lim_{t \to \infty} u(t)\) given \(u(0) = 1\).  
      \blanklines{5}
    \item Determine the long-term behaviour of \(u(t)\) given \(u(43738921740321678914) = 1\).  
      \blanklines{5}
    \item Evaluate \(\lim_{t \to \infty} u(t)\) given \(u(-4326) = -0.723\).  
      \blanklines{5}
    \item Determine the long-term behaviour of \(u(t)\) given \(u(-1) = 6\).  
      \blanklines{5}
    \item Suppose \(u(\pi) = 5.0001\). After a long time, what do we expect \(u(t)\) to approach?
      \blanklines{5}
  \end{enumerate}
\end{example}

% \faStar{} We \hlsupp{do not need to solve} \emph{autonomous} differential equations to understand \hlmain{how their equilibria affect solutions}.  The idea we need is a slight change of mindset: Treat the dependent variable \(u\) in a differential equation as the independent variable for \(u'\). 

% \begin{definition}[phase-line plots] \label{def:phase-line-plots}
%   Suppose \(u' = g(u)\) is an autonomous differential equation where \(g\) is continuous. The (unknown) dependent variable is \(u\).
%
%   The \hlmain{phase-line plot} of a differential equation is the graph of \(u' = g(u)\) where the vertical axis is \(u'\) and horizontal axis is \(u\).  
%
%   We \hlsupp{do not need to solve} the differential equation at all.
% \end{definition}

\clearpage


\begin{example}
  Find the long-term behaviour of the initial-value problem 
  \[
    u'(t) = (u(t) - 3) (2u(t) - 1) (u(t) - 5) \quad\text{and}\quad u(1) = 4.
  \]
  \blanklines{10}
\end{example}


\begin{example}
  A population \(N(t)\) is modelled by 
  \[
    N'(t) = N(t)^{2} + N(t) - 6 \quad\text{and}\quad u(500) = 2.99.
  \]
  Predict the number to which the population approaches after a long time.
  \blanklines{10}
\end{example}

\begin{example}
  Consider the following phase-line plot of an autonomous differential equation.  Identify its equilibria and their stability and fill in the blanks with suitable inequalities.
  \begin{itemize}
    \item If \(\lim_{t \to \infty} u(t) = 3\), then \(\underline{\hspace{1in}} u(\text{0.5}) \underline{\hspace{1in}}\). 
    \item If \(\lim_{t \to \infty} u(t) = \infty\), then \(\underline{\hspace{1in}} u(\text{3}) \underline{\hspace{1in}}\). 
    \item If \(\lim_{t \to \infty} u(t) = -\infty\), then \(\underline{\hspace{1in}}  u(\text{4})  \underline{\hspace{1in}}\). 
  \end{itemize}

  \includegraphics{../standalones/build/plot-phase-lines-example2}
\end{example}


\clearpage
\begin{example}
  Consider the differential equation
  \[
    N'(t) = (N(t) - 1) (N(t) - 3) (N(t) - 20)^{3}.
  \]

  Which of the following initial values lead to \(\lim_{t \to \infty} u(t) = 3\)? In other words, the question asks that if \(u(t)\) is a solution to the differential equation, which initial condition \(u(t)\) must satisfy so that \(\lim_{t \to \infty} u(t) = 3\)?

  There are multiple correct answers.
  \begin{enumerate}[label=(\alph*)]
    \item \(u(0) = 19\)
    \item \(u(-1) = 21\)
    \item \(u(3) = 0.9\)
    \item \(u(5) = 1.1\)
    \item \(u(2000) = 2.5\)
  \end{enumerate}
\end{example}

\bigskip
The next exercise asks you to combine stability analysis and your knowledge from Calculus 1000 to deduce phase-line plots of near an equilibrium that is neither stable nor unstable. For hints, see page 67 of the textbook.

\begin{example}
  In the plots below, sketch the phase-line plots of four differential equations such that \(g'(\bar{u}) = 0\), but two are stable, and two are neither stable nor unstable.

  \begin{center}
    \includegraphics{../standalones/build/plot-phase-lines-stability}
    \quad
    \includegraphics{../standalones/build/plot-phase-lines-stability}
    \quad
    \includegraphics{../standalones/build/plot-phase-lines-stability}
    \quad
    \includegraphics{../standalones/build/plot-phase-lines-stability}
  \end{center}
\end{example}

\end{document}
