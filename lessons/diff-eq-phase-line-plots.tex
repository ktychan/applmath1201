%! TeX program = lualatex
\documentclass[../main.tex]{subfiles}
\begin{document} \section{Phase-line plots}

Getting an exact solution to a differential equation could be more challenging than worth our investment.  For autonomous equations, phase-line plots allow us to \hlattn{visualize} \hlmain{inexact solutions} of autonomous differential equations and their \hlmain{qualitative features}. 

It's important to remember (from the very first lecture of the semester) that models are \emph{imperfect} representations of natural phenomenons. Therefore, exact solution to differential 

\begin{definition}[phase-line plots]
  Suppose \(u' = g(u)\) is a (autonomous) differential equation where \(g\) is continuous. The unknown function is \(u(t)\). 

  The graph of \(u' = g(u)\) where we treat \(u\) as the independent variable and \(u'\) as the dependent variable is the phase-line plot of the differential equation.
\end{definition}

\faStar{} Equilibrium and their stability properties can be read off from phase-line plots. 

\begin{example}
  Let's reconsider the logistic equation \(N' = 2\left(1 - \frac{N}{10}\right)N\) with \(N(0) > 0\). Pretend for a moment that we don't know how to solve it, and try to use phase-line plots to understand it qualitatively.

  \begin{figure}[H] % [h] for here, [ht] for here top, [hb] for here bottom
    \centering
    \includegraphics{../standalones/build/plot-phase-lines-logistic.pdf}
    \caption{Phase-line plot of \(N' = 2\left(1-\frac{N}{10}\right)N\).}
    \label{fig:phase-lines-logistic}
  \end{figure}

  \begin{enumerate}
    \item Recall that \(N\) is a function of \(t\). Pick a number on the horizontal axis, say \(N = 2\). 

      Note that \(N = 2\) represents all solutions (plural!) to the given logistic equation that intersect the horizontal line \(N = 2\) in the \(t\) vs \(N\) plot.
  \end{enumerate}
\end{example}
\end{document}
