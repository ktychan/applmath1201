%! TeX program = lualatex
\documentclass[../main.tex]{subfiles}
\begin{document} \section{Equilibria and their stability}

Recall (from the very first lecture of the semester) that models are \emph{imperfect} representations of natural phenomena. Exact solutions to differential equations may or may not be what we \emph{need}.  

{\footnotesize Fun read: \url{https://www.jpl.nasa.gov/edu/news/how-many-decimals-of-pi-do-we-really-need/}}

We are about to introduce phase-line plots. Feel free to preview Definition~\ref{def:phase-line-plots} before working through the next example.

\begin{example}[equilibria of logistic equations]
  Solutions of logistic equations exhibit \emph{informative} patterns.
  \begin{figure}[H] % [h] for here, [ht] for here top, [hb] for here bottom
    \centering
    \includegraphics{../standalones/build/plot-logistic-solutions}
    \label{fig:logistic-solutions}
  \end{figure}
  
  \blanklines{10}
\end{example}

\faStar{} We \hlsupp{do not need to solve} \emph{autonomous} differential equations to understand \hlmain{how their equilibria affect solutions}.  The idea we need is a slight change of mindset: Treat the dependent variable \(u\) in a differential equation as the independent variable for \(u'\). 

\begin{definition}[phase-line plots] \label{def:phase-line-plots}
  Suppose \(u' = g(u)\) is an autonomous differential equation where \(g\) is continuous. The (unknown) dependent variable is \(u\).

  The \hlmain{phase-line plot} of a differential equation is the graph of \(u' = g(u)\) where the vertical axis is \(u'\) and horizontal axis is \(u\).  

  We \hlsupp{do not need to solve} the differential equation at all.
\end{definition}

Equilibria and their stability properties (later) can be read off from phase-line plots. 

\end{document}
