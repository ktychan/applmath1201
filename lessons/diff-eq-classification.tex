%! TeX program = lualatex
\documentclass[../main.tex]{subfiles}
\begin{document} \section{Classifying differential equations}

We group differential equations by certain characteristics in the \emph{dependent} variable.

\begin{definition}[order]
The \hlmain{order} of a differential equation is the order of the highest derivative of the dependent variable in the equation.
\end{definition}
\begin{example}
  Identify the order of the following differential equations.
  \[
    \frac{du}{dt} + u^{2} = t^{3}, 
    \qquad
    \left(u(t)''\right)^{3} + \sin(t) = 0.
  \]
  \blanklines{5}
\end{example}

\begin{definition}[linear vs non-linear]
  If a differential equation has no product of the dependent variable with itself or any of its derivatives, then the equation is called \hlmain{linear}; otherwise, it is called \hlmain{non-linear}. 
\end{definition}
\begin{example}
  Find the linear differential equation from the list below.
  \[
    u(t) + t^{2} u'(t) = 0,
    \qquad
    u'(t) + u'(t) u(t) = 0,
    \qquad
    \big(u(t)\big)^{2} + u'(t) u^{(5)}(t) = t u(t).
  \]
  \blanklines{5}
\end{example}

\begin{definition}[linear, first-order]
  Assume \(u\) is a function of \(t\). A \hlmain{linear, first-order} differential equations is a differential equation of the form
  \[
    u' + p(t) u = q(t), \quad\text{where \(p(t)\) and \(q(t)\) are functions of \(t\)}.
  \]
\end{definition}

\faExclamationTriangle{} The definition and notation of linear, first-order differential equation says that the dependent variable \(u\) does not appear in \(p(t)\) and \(q(t)\).

Lastly, an \hlmain{autonomous equation} is a differential equation in which its independent variable does not appear at all. 
\blanklines{5}
\end{document}
