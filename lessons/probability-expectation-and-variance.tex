%! TeX program = lualatex
\documentclass[../main.tex]{subfiles}
\begin{document} \section{Expectation and variance}

Probability gives a big picture view of data. Pictorially, probability mass/density functions let us visualize the data. Other common ways of thinking about data involves computing their \hlmain{expectation} and \hlmain{variance}. 

\begin{definition}[expectation]
  The \hlmain{expectation} of a discrete random variable \(X\) with PMF \(f_{X}(x)\) and values \(x_{1}, \ldots, x_{n}\) is
  \begin{equation} \label{eq:expectation-discrete}
    E(X) = \sum_{i = 1}^{n} x_{i} f_{X}(x_{i}).
  \end{equation}

  The \hlmain{expectation} of a continuous random variable \(X\) with PDF \(f(x)\) is
  \begin{equation} \label{eq:expectation-continuous}
    E(X) = \int_{-\infty}^{\infty} x f(x) \;dx.
  \end{equation}
\end{definition}

\begin{example} \label{ex:variance-discrete-betting}
  We play a betting game of predicting the outcome of rolling a fair six-sided die.  Each bet costs \(1\) chore token. If we win, we collect \(3\) chore tokens.

  This game is modelled by the random variable \(X\) where \(X = -1\) is the event in which we lose and \(X = 3\) is the event in which we win. We work out that \(\mathbb{P}(X = -1) = \underline{\hspace{1cm}}\) and \(\mathbb{P}(X = 3) = \underline{\hspace{1cm}}\).  What is the expected return if we play the game \(10\) times?

  \blanklines{10}
\end{example}

\begin{example} \label{ex:variance-continuous-uniform}
  Let \(X\) be a continuous random variable with PDF \(f(x) = 1/5\) on \([-1, 1]\), \(f(x) = 1/10\) on \([2,8]\) and \(0\) everywhere else.  Calculate \(E(X)\).

  \blanklines{15}
\end{example}
\clearpage

Variance is defined as \(Var(X) = E( (X - E(x))^{2} )\) regardless of whether a random variable \(X\) is discrete or continuous. However, this formula needs further explanation because it does not \emph{directly} tell us how to calculate variance.
\begin{definition}[variance and standard deviation]
  Let \(\mu = E(x)\) which is a constant.

  The \hlmain{variance} of a discrete random variable \(X\) with PMF \(f_{X}(x)\) and values \(x_{1}, \ldots, x_{n}\) is 
  \begin{equation} \label{def:variance-discrete}
    Var(X) = \sum_{i=1}^{n} (x_{i} - \mu)^{2} f_{X}(x_{i}).
  \end{equation}

  The \hlmain{variance} of a continuous random variable \(X\) with PDF \(f(x)\) is 
  \begin{equation} \label{def:variance-continuous}
    Var(X) = \int_{-\infty}^{\infty} (x - \mu)^{2} f(x) \;dx.
  \end{equation}

  The \hlmain{standard deviation} (for both types) is \(Stddev(X) = \sqrt{Var(X)}\).

  It is quite common to denote standard deviation by \(\sigma\) or \(s\) and variance by \(\sigma^{2}\) or \(s^{2}\).
\end{definition}

\begin{example}
  Calculate the variance for the discrete random variable in Example~\ref{ex:variance-discrete-betting}.

  \blanklines{10}
\end{example}

\begin{example}
  Calculate the variance for the continuous random variable in Example~\ref{ex:variance-continuous-uniform}.

  \blanklines{15}
\end{example}
\clearpage

\faStar{} Expectation measures the \hlmain{centre} of data. Variance and standard deviation measure how far the data spread from its expectation.

\begin{example} \label{ex:drv-shapes}
  Consider the following PMFs.

  \begin{tikzpicture}
    \begin{axis}[width=3in, height=2in, xmin=0, xmax=9, ymin=0, ymax=1, axis y line=none, axis lines=middle, no markers, minor tick num=1, title={\(\mu = 3.75, \sigma \approx 2.7581\)}]
      \addplot[main] coordinates { (1,0) (1,4/16) } node[above] {\footnotesize \color{black} \(4/16\)}; \node at (axis cs:1,4/16) {\(\bullet\)};
      \addplot[main] coordinates { (2,0) (2,4/16) }; \node at (axis cs:2,4/16) {\(\bullet\)};
      \addplot[main] coordinates { (4,0) (4,4/16) }; \node at (axis cs:4,4/16) {\(\bullet\)};
      \addplot[main] coordinates { (8,0) (8,4/16) }; \node at (axis cs:8,4/16) {\(\bullet\)};
    \end{axis}
  \end{tikzpicture}
  \begin{tikzpicture}
    \begin{axis}[width=3in, height=2in, xmin=0, xmax=9, ymin=0, ymax=1, axis y line=none, axis lines=middle, no markers, minor tick num=1, title={\(\mu \approx 2.68, \sigma \approx 2.7593\)}]
      \addplot[main] coordinates { (1,0) (1,5/16) } node[above] {\footnotesize \color{black} \(5/16\)}; \node at (axis cs:1,5/16) {\(\bullet\)};
      \addplot[main] coordinates { (2,0) (2,5/16) }; \node at (axis cs:2,5/16) {\(\bullet\)};
      \addplot[main] coordinates { (4,0) (4,5/16) }; \node at (axis cs:4,5/16) {\(\bullet\)};
      \addplot[main] coordinates { (8,0) (8,1/16) } node[above] {\footnotesize \color{black} \(1/16\)}; \node at (axis cs:8,1/16) {\(\bullet\)};
    \end{axis}
  \end{tikzpicture}
  \begin{tikzpicture}
    \begin{axis}[width=3in, height=2in, xmin=0, xmax=9, ymin=0, ymax=1, axis y line=none, axis lines=middle, no markers, minor tick num=1, title={\(\mu \approx 3.88, \sigma \approx 2.7629\)}]
      \addplot[main] coordinates { (1,0) (1, 2/16) } node[above] {\footnotesize \color{black} \(2/16\)}; \node at (axis cs:1, 2/16) {\(\bullet\)};
      \addplot[main] coordinates { (2,0) (2, 2/16) }; \node at (axis cs:2, 2/16) {\(\bullet\)};
      \addplot[main] coordinates { (4,0) (4,10/16) } node[above] {\footnotesize \color{black} \(10/16\)}; \node at (axis cs:4,10/16) {\(\bullet\)};
      \addplot[main] coordinates { (8,0) (8, 2/16) }; \node at (axis cs:8, 2/16) {\(\bullet\)};
    \end{axis}
  \end{tikzpicture}

  \blanklines{2}

  \begin{tikzpicture}
    \begin{axis}[width=3in, height=2in, xmin=0, xmax=9, ymin=0, ymax=1, axis y line=none, axis lines=middle, no markers, minor tick num=1, title={\(\mu = 3.00, \sigma \approx 1.8945\)}]
      \addplot[main] coordinates { (1,0) (1,4/16) } node[above] {\footnotesize \color{black} \(4/16\)}; \node at (axis cs:1,4/16) {\(\bullet\)};
      \addplot[main] coordinates { (2,0) (2,4/16) }; \node at (axis cs:2,4/16) {\(\bullet\)};
      \addplot[main] coordinates { (4,0) (4,4/16) }; \node at (axis cs:4,4/16) {\(\bullet\)};
      \addplot[main] coordinates { (5,0) (5,4/16) }; \node at (axis cs:5,4/16) {\(\bullet\)};
    \end{axis}
  \end{tikzpicture}
  \begin{tikzpicture}
    \begin{axis}[width=3in, height=2in, xmin=0, xmax=9, ymin=0, ymax=1, axis y line=none, axis lines=middle, no markers, minor tick num=1, title={\(\mu = 2.50, \sigma \approx 1.8963\)}]
      \addplot[main] coordinates { (1,0) (1,5/16) } node[above] {\footnotesize \color{black} \(5/16\)}; \node at (axis cs:1,5/16) {\(\bullet\)};
      \addplot[main] coordinates { (2,0) (2,5/16) }; \node at (axis cs:2,5/16) {\(\bullet\)};
      \addplot[main] coordinates { (4,0) (4,5/16) }; \node at (axis cs:4,5/16) {\(\bullet\)};
      \addplot[main] coordinates { (5,0) (5,1/16) } node[above] {\footnotesize \(1/16\)}; \node at (axis cs:5,1/16) {\(\bullet\)};
    \end{axis}
  \end{tikzpicture}
  \begin{tikzpicture}
    \begin{axis}[width=3in, height=2in, xmin=0, xmax=9, ymin=0, ymax=1, axis y line=none, axis lines=middle, no markers, minor tick num=1, title={\(\mu = 3.50, \sigma \approx 1.9016\)}]
      \addplot[main] coordinates { (1,0) (1, 2/16) } node[above] {\footnotesize \color{black} \(2/16\)}; \node at (axis cs:1, 2/16) {\(\bullet\)};
      \addplot[main] coordinates { (2,0) (2, 2/16) }; \node at (axis cs:2, 2/16) {\(\bullet\)};
      \addplot[main] coordinates { (4,0) (4,10/16) } node[above] {\footnotesize \color{black} \(10/16\)}; \node at (axis cs:4,10/16) {\(\bullet\)};
      \addplot[main] coordinates { (5,0) (5, 2/16) }; \node at (axis cs:5, 2/16) {\(\bullet\)};
    \end{axis}
  \end{tikzpicture}
\end{example}

\begin{example} \label{ex:crv-shapes}
  Consider the following PDFs.

  \begin{center}
    \begin{tikzpicture}
      \begin{axis}[width=6in, height=2in, smooth, xmin=-5, xmax=5, ymin=0, ymax=1, no markers, smooth, samples=100, axis lines=middle, ytick={0,1}, xtick={-4,-2,0,2,4}, minor tick num=1]
        \addplot[very thick, main] {normalpdf(x, 0, 2/2)};
        \addplot[very thick, supp] {normalpdf(x, 0, 1/2)};
      \end{axis}
    \end{tikzpicture}

    \begin{tikzpicture}
      \begin{axis}[width=6in, height=2in, smooth, xmin=-5, xmax=5, ymin=0, ymax=1, no markers, smooth, samples=100, axis lines=middle, ytick={0,1}, xtick={-4,-2,0,2,4}, minor tick num=1]
        \addplot[very thick, main] {normalpdf(x,    1, 1/2)};
        \addplot[very thick, supp] {normalpdf(x, -1.5, 1/2)};
      \end{axis}
    \end{tikzpicture}
  \end{center}
\end{example}
\clearpage

Examples~\ref{ex:drv-shapes}~and~\ref{ex:crv-shapes} demonstrate the general observation that \hlmain{expectation \(\mu\), variance \(\sigma^{2}\) and standard deviation \(\sigma\)} together roughly describe the \hlmain{shape of a random variable}.  Sometimes (in reality and in this course), we want to randomly generate data that replicate experiments with certain sample mean and sample variance.  Look ahead to Example~\ref{ex:simulation} on page~\pageref{ex:simulation} for a preview.

In mathematical terms, we want to find a suitable PDF given its expectation \(\mu\) and variance \(\sigma^{2}\) (or standard deviation \(\sigma\)). The idea (not the computational steps) we used in Example~\ref{ex:distribution-with-unknown-bounds} is helpful.

\begin{example} \label{ex:uniform-distribution-from-mean-and-stddev}
  We want a continuous random variable \(X\) with 
  \[
    \text{PDF }
    f(x) = 
    \begin{cases}
      \frac{1}{2a} & \text{if } |x| < a \\
      0 & \text{otherwise}
    \end{cases},
    \quad
    \mu = 0 \quad\text{and}\quad \sigma^{2} = \frac{17.25}{11}.
  \]
  Find the unknown parameter \(a\). Approximate to four decimal places if necessary.

  \blanklines{20}
\end{example}
\clearpage

\end{document}
