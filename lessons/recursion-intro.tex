%! TeX program = lualatex
\documentclass[../main.tex]{subfiles}
\begin{document} \section{Introduction to discrete-time models and recursion}

\hlmain{Discrete-time models} are models in which the time variable takes on (typically non-negative) integer values. In discrete-time models, time progresses in discrete steps instead of flowing continuously. 

\hlmain{Recursions} are equations for describing changes over discrete time steps and are often used to describe discrete-time models.

We introduce these two ideas together through Example~\ref{ex:origami}.  

\begin{example} \label{ex:origami}
  Repeatedly fold a triangular paper by identifying the two corners on the long edge. 
  \begin{figure}[H]
    \centering
    \includegraphics{../standalones/build/demo-origami}
  \end{figure}

  We wish to count the number of triangles on the (unfolded) paper after \(t\) folds. 

  \faStar{} Below are the typical steps for developing discrete-time models using recurrence.  Later this week, we will learn some mathematical methods to get information from recurrence.

  \begin{enumerate}[wide]
    \item \hlsupp{Set up notations}. Use a \emph{complete sentence} such as ``let \ldots{} be \ldots'' to attach meaning to symbols.
      \blanklines{5}
    \item \hlsupp{Describe the initial condition}. Describe the number of triangles after \(0\) fold.
      \blanklines{5}
    \item \hlsupp{Describe change}. Describe the \hlmain{change} in triangles from \(t-1\) fold to \(t\) fold \emph{using an equation}.
      \blanklines{15}
  \end{enumerate}
\end{example}

Let's sharpen the concept of recurrence by writing down a precise definition. 

\begin{definition}[recurrence]
  Let \(u_{t}\) be discrete-time variables, i.e., time \(t\) are non-negative integers.  We call \(u_{t}\) the \hlmain{state at time \(t\)}. 

  A \hlmain{recurrence} has two parts:
  \begin{enumerate}
    \item An \hlmain{initial condition} (or initial state) \(u_{0}\).
    \item A \hlmain{recurrence equation} that relate \(u_{t}\) to one or more \emph{previous} states.
  \end{enumerate}

  \faExclamationTriangle{} Recurrences are used to describe discrete-time models.
\end{definition}

\begin{example}
  Here are some examples of recurrences. For each recurrence below, identify its initial condition and its recurrence equation.

  \begin{enumerate}[wide, itemsep={2ex}]
    \item 
      A recurrence is commonly written in a single line.
      \begin{equation} \label{eq:recurrence-intro-oneline}
        p_{0} = 1 \quad\text{and}\quad p_{t} = \frac{1}{2} p_{t-1} - \frac{1}{2}, \text{ for integers } t \ge 1.
      \end{equation}

    \item 
      A recurrence can be written like a piecewise function. The left brace emphasizes that the initial condition and the recurrence equation should be thought of together. 
      \begin{equation} \label{eq:recurrence-intro-left-brace}
        \left\{
          \begin{aligned}
            a_{0} &= 3,  \\
            a_{t} &= a_{t-1} + 2,  \text{ for integers } t \ge 1.
          \end{aligned}
        \right.
      \end{equation}

    \item 
      A recurrence can involve any algebraic operations.
      \begin{equation} \label{eq:recurrence-intro-sqrt}
        b_{0} = 4 \quad\text{and}\quad b_{t} = t \sqrt{b_{t-1}}, \text{ for integers } t \ge 1.
      \end{equation}

    \item 
      The initial state of a recurrence can be hard to measure, deliberately left as unknown, or simply unimportant. In any case, an initial condition \emph{always exists} but can be unspecified.  

      \begin{equation} \label{eq:recurrence-intro-unspecified-initial-condition}
        f_{t} = \frac{1}{f_{t-1}} + 1.
      \end{equation}

      A recurrence equation \hlsupp{without an initial state} represents \underline{\hfill{} \phantom{Xy infinitely many recurrences Xy Xy}}, one for every possible initial condition. Equation~\eqref{eq:recurrence-intro-unspecified-initial-condition} is such an example.

    \item 
      A recurrence can involve one or more previous states.
      \begin{equation} \label{eq:recurrence-intro-fib}
        F_{t} = F_{t-1} + F_{t-2}.
      \end{equation}
  \end{enumerate}
\end{example}
\end{document}
