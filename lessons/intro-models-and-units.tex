%! TeX program = lualatex
\documentclass[../main.tex]{subfiles}
\begin{document} \section{Sanity checks for models}
Example~\ref{ex:toucan} teaches us two important ideas in modelling.
\begin{enumerate}
  \item A model should make sense physically. We do so by checking that the units on both sides of the equation match.

  \item Visual inspection is often an inaccurate method of evaluation for accuracy. We often need to transform our data to discover an accurate model.
\end{enumerate}

We now study both of the above ideas in more details.

\begin{mdframed}[style=simple]
  \textbf{Idea 1}. A model, expressed as a mathematical equation, is \hlmain{dimensionally homogeneous} if the units associated with each of its terms match.
\end{mdframed}
\faPencil*{} ``Dimensionally homogeneous'' is also known as ``dimensionally consistent'' (often in physics).

\begin{example}
  It is often said that ``time is money.'' Expressed as a mathematical model, we have 
  \[
    t = M
  \]
  where \(t\) is time (minutes) and \(M\) is money (dollars).

  Is this model dimensionally homogeneous?
  \blanklines{8}
\end{example}

\begin{example}
  Einstein's model \(E = mc^{2}\) is dimensionally homogeneous. The symbol \(E\) represents energy whose unit is \(\text{kg} \cdot \text{m}^{-2} \cdot \text{s}^{-2}\). The symbol \(m\) is mass whose unit is kg. 

  What is the unit of \(c\)?
  \blanklines{10}
\end{example}
\clearpage

\todo{Depending on how much we need students to understand log-log space, we might want to add a page on log-log space here.}

\begin{mdframed}[style=simple]
  \textbf{Idea 2}. Suppose we have a data set with a best-fit model \(y = c x^{\alpha}\). 
  The log-log transformation of such a data set, a plot \(\ln(x)\) against \(\ln(y)\), is a straight line whose slope is \(\alpha\) and \(y\)-intercept is \(\ln(c)\).

  Conversely, in the log-log space, if a straight line \(\ln(y) = y_{0} + \alpha \ln(x)\) is the best fit of a log-log transformed data set, then the model \(y = e^{y_{0}} x^{\alpha}\) fits the original data best.
\end{mdframed}

\begin{example}
  Find the best-fit model for the following data in log-log space.

  \todo{TODO: make a plot}.
\end{example}

\begin{example}
  Sketch the model \(y = 2 \sqrt{x}\) in the log-log space.
  
  \todo{TODO: make a blank log-log plot}.
\end{example}

\clearpage
In Example~\ref{ex:toucan}, we used the relation \[\text{weight} = (\text{density}) \times (\text{weight}) \] to guide our choice of a model. 

\begin{mdframed}[style=simple]
  If two quantities \(x,y\) are related by a \hlattn{linear} relation \(y = c x\) for some constant \(c > 0\), then we say \hlmain{\(x\) and \(y\) exhibit isometry}.

  If two quantities \(x,y\) are related by a \hlattn{non-linear} relation, i.e., \(y = c x^{\alpha}\) where \(\alpha \ne 1\), then we say \hlmain{\(x\) and \(y\) exhibit allometry}.
\end{mdframed}

\begin{example}
  Consider the length, volume and weight of birds as discussed in Example~\ref{ex:toucan}.

  \begin{enumerate}[wide]
    \item Do length and weight exhibit isometry or allometry? Explain your reasoning.
      \blanklines{10}

    \item Do volume and weight exhibit isometry or allometry? Explain your reasoning.
      \blanklines{10}
  \end{enumerate}
\end{example}
\end{document}
