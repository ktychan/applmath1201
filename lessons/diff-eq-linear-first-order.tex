%! TeX program = lualatex
\documentclass[../main.tex]{subfiles}
\begin{document} \section{Solving linear, first-order differential equations}

Linear, first-order differential equations are (relatively) easy to solve.

\begin{thm} \label{thm:diff-eq-linear-first-order}
  We wish to solve for the unknown function \(u(t)\) in
  \begin{equation} \label{eq:diff-eq-linear-first-order}
    u'(t) + p(t) u(t) = q(t).
  \end{equation}
  
  Define
  \begin{equation} \label{eq:diff-eq-integrating-factor}
    v(t) = e^{\int p(t) \;dt} \text{ and choose \(C = 0\) for convenience}.
  \end{equation} 

  The general solution to Equation~\eqref{eq:diff-eq-linear-first-order} is
  \begin{equation} \label{eq:diff-eq-linear-first-order-solution1}
    u(t) = \frac{1}{v(t)} \left( \int v(t) q(t) \;dt + C \right).
  \end{equation}
\end{thm}

\faExclamationTriangle{} Make \hlattn{exactly one} of the choices below to interpret Equation~\eqref{eq:diff-eq-linear-first-order-solution1} to avoid confusion.  \hlmain{Both choices are valid.} Pick whichever resonates with you.

\begin{enumerate}
  \item You expect yourself to forget to add a ``\(+C\)'' after evaluating an indefinite integral.

    In such a case, remember Equation~\eqref{eq:diff-eq-linear-first-order-solution1} as stated above and omit the integration constant after evaluating \(\textstyle \int v(t) q(t) \;dt\).  Leaving the ``\(+C\)'' in the equation serves as a reminder that an integration constant should be there.

  \item You expect yourself to always include a ``\(+C\)'' after evaluating an indefinite integral.

    In such a case, remember Equation~\eqref{eq:diff-eq-linear-first-order-solution1} as \(u(t) = \frac{1}{v(t)}\left( \int v(t) q(t) \;dt \right)\) because you don't need an extra reminder that an indefinite integral represents infinitely many antiderivatives (of the integrand).

    You should also remember that \(\textstyle \int 0 \;dt = C\).
\end{enumerate}

\begin{example}[make your choice]
  Use Theorem~\ref{thm:diff-eq-linear-first-order} to find the general solution for \(u'(t) + u(t) = 1\).
  {\footnotesize The answer is \(u(t) = 1 + C e^{-t}\) where \(C\) is a general constant.}
  \blanklines{12}
\end{example}

\faStar{} Equations~\eqref{eq:diff-eq-integrating-factor} and \eqref{eq:diff-eq-linear-first-order-solution1} both come out of nowhere, and Equation~\eqref{eq:diff-eq-linear-first-order-solution1} looks rather complicated. The good news is that Theorem~\ref{thm:diff-eq-linear-first-order} is the (overly?) compact form of Method~\ref{method:integrating-factor} on page~\pageref{method:integrating-factor}.

\clearpage

\begin{method}[the integrating factor] \label{method:integrating-factor}
  \hlmain{An integrating factor} for a linear, first-order differential equation \(u'(t) + p(t) u(t) = q(t)\) is
  \begin{equation} 
    \color{main}
    v(t) = e^{\int p(t) \;dt} \text{ and choose \(C = 0\) for convenience}. \tag{\ref{eq:diff-eq-integrating-factor}}
  \end{equation}
  To find the general solution for the differential equation, multiply through by the integrating factor \(v(t)\) and reorganize the \hlattn{left-hand side as the result of the product rule}. The general solution of the differential equation, Equation~\eqref{eq:diff-eq-linear-first-order-solution1}, is the result of \hlattn{integrating both sides} then \hlattn{solving for \(u(t)\)}.
\end{method}

\faExclamationTriangle{} If a linear, first-order differential equation is not written in the form \(u'(t) + p(t) u(t) = q(t)\), then you must manipulate it to this form before applying the integrating factor method or Theorem~\ref{thm:diff-eq-linear-first-order}.

\begin{example} \label{ex:diff-eq-linear-first-order-1}
  Use Method~\ref{method:integrating-factor} to find the general solution for \( u'(t) + \tfrac{2}{t} u(t) = t^{2}\).
  
  \begin{enumerate}[wide, label=(Step~\arabic*)]
    \item Multiply through by an integrating factor \(v(t)\).
      \blanklines{10}

    \item Rewrite the left-hand side as \(\frac{d}{dt} \bigg( u(t) v(t) \bigg)\).  This is the clever idea.
      \blanklines{5}
      
    \item Integrate both sides and solve for \(u(t)\). \faExclamationTriangle{} To minimize mistakes, evaluate both integrals at the same time and keep only the integration constant on the right-hand side for convenience. 
      \blanklines{15}
  \end{enumerate}
\end{example}
\clearpage

Solving initial-value problems requires us to combine the general solution and the initial condition to solve for \(C\).  As demonstrated in Example~\ref{ex:diff-eq-IVP-calc-1000} on page~\pageref{ex:diff-eq-IVP-calc-1000}, we have worked with simpler versions of initial-value problems in Calculus~1000.

\begin{example} \label{ex:diff-eq-linear-first-first-order-1-ivp}
  Continue from Example~\ref{ex:diff-eq-linear-first-order-1}. Solve the initial-value problem 
  \[
    u'(t) + \frac{2}{t} u(t) = t^{2} \quad\text{and}\quad u(1) = 2.
  \]
  \blanklines{10}
\end{example}

Here are two more exercises using the same idea.
\begin{example}
  Continue from Example~\ref{ex:diff-eq-linear-first-order-1}. Solve the initial-value problem 
  \[
    u'(t) + \frac{2}{t} u(t) = t^{2} \quad\text{and}\quad u(10) = \sqrt{2}.
  \]
  \blanklines{10}
\end{example}

\begin{example}
  Continue from Example~\ref{ex:diff-eq-linear-first-order-1}. Solve the initial-value problem 
  \[
    u'(t) + \frac{2}{t} u(t) = t^{2} \quad\text{and}\quad u(3) = 0.
  \]
  \blanklines{10}
\end{example}
\clearpage

Here is a comprehensive initial-value problem.
\begin{example}
  Solve the initial-value problem
  \[
    u'(t) - \frac{2}{t+1} u(t) = t \quad\text{and}\quad u(0) = \pi.
  \]
  The solution is longer than that of Calculus~1000 problems and is typical of such a type of problems. 
  \blanklines{49}
\end{example}
\clearpage

On this page, we look at a subclass of linear, first-order differential equations. A linear, first-order differential equation \(u'(t) + p(t) u(t) = q(t)\) is called \hlmain{homogeneous} if \(q(t) = 0\); otherwise, it is called \hlmain{inhomogeneous}.  

Homogeneous differential equations \(u'(t) + p(t) u(t) = 0\) can be solved using Method~\ref{method:integrating-factor} or just the good'ol substitution rule.  In other words, if you run into a homogeneous equation, feel free to think ``\emph{ah! I can use the substitution rule if I wanted to}.''  The only question is ``\emph{how?}''.

\begin{example}
  Find the general solution for \(u'(t) - u(t) = 0\).

  Notice this is a \emph{homogeneous} equation. Trying using the substitution rule. If you carefully deal with absolute values and analyze the values of constants, then you should get \(u(t) = A e^{t}\) where \(A\) can be \emph{any} constant.

  \blanklines{15}
\end{example}

\begin{example}
  Solve the initial-value problem \(y'(x) + \frac{y(x)}{x^{2} + 1} = 0\) and \(y(0) = 1\).  Why not try both methods and compare the pros and cons? The solution is \(y(x) = e^{-\arctan(x)}\).

  \blanklines{20}
\end{example}
\clearpage

In real-life applications (and exams), we often wish to understand the long-term behaviour of a differential equation.
\begin{example}
  The mass of a radioactive\footnote{Fairy dust is probably radioactive because they seem to lose their magic exponentially quickly.}fairy dust \(u(t)\) satisfies 
  \[
    e^{t}u'(t) - u(t) - 1 = 0 \text{ with an unknown mass at \(t = 0\)}.
  \]
  Predict the mass of the fairy dust as time approaches \(\infty\). Assume mass is measured in grams.
  
  \blanklines{47}
\end{example}

Here is one more example that combines differential equations and calculus. 
\begin{example}
  Find \(\lim_{x \to \infty} y(x)\) given that \(y(x)\) satisfies
  \[
    \frac{dy}{dx} + y = e^{x} \text{ and \(y(0) = \frac{3}{2}\)}.
  \]
  \blanklines{45}
\end{example}
\clearpage

Here are some generic practice problems.
\begin{example}
  Find the general solution for \(\frac{du}{dt} + u(t) = t^{2}e^{-t}\).
  \blanklines{50}
\end{example}
\clearpage

\begin{example}
  Solve the initial-value problem \(u'(t) + tu(t) = 3t\) with \(u(0) = 1\).  

  What is the qualitative behaviour (increase, no change or decrease) of the solution function at \(t = 1\)?  

  How about at \(t = 0\)? Is there a really quick way to answer this question without solving for \(u(t)\)?
  \blanklines{50}
\end{example}
\clearpage
\end{document}
