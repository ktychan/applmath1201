%! TeX program = lualatex
\documentclass[../main.tex]{subfiles}
\begin{document} \section{Homogeneous linear systems of DEs with distinct eigenvalues}
  
In this section, we solve systems of differential equations 
\begin{equation} \label{eq:homogeneous-linear-DE}
  \frac{d\vec{x}}{dt} = A \vec{x}(t) \quad\text{with initial condition}\quad \vec{x}(0).
\end{equation}

\begin{method}[solutions of homogeneous {\(2\)-by-\(2\)} systems of DEs with \hlattn{distinct} eigenvalues] \label{method:DE-homogeneous-linear-distinct-eigenvalues}
  Find eigenvectors \(\vec{u}\) with eigenvalue \(\mu\) and \(\vec{v}\) with eigenvalue \(\lambda\).

  If \hlattn{\(\mu \ne \lambda\)}, then the following applies; otherwise, use Method~\ref{method:DE-homogeneous-linear-rank-1}.

  Solve for scalars \hlsupp{\(\alpha, \beta\)} so that \(\vec{x}(0) = \hlsupp{\alpha} \vec{u} + \hlsupp{\beta} \vec{v}\). The solution to Equation~\eqref{eq:homogeneous-linear-DE} is
  \begin{equation} \label{eq:DE-homogeneous-linear-distinct-eigenvalues}
    \vec{x}(t) = \hlsupp{\alpha} \cdot \hlmain{\left( \parbox[c]{1.6in}{\centering fundamental solution associated to \(\vec{u}\)} \right) } + \hlsupp{\beta} \cdot \hlmain{\left( \parbox[c]{1.6in}{\centering fundamental solution associated to \(\vec{v}\)} \right)}.
  \end{equation}

  The \hlmain{fundamental solution} associated to an eigenvector \(\vec{v}\) with eigenvalue \(\lambda\) is \hlmain{\(e^{\lambda t} \vec{v}\)}.  However, if \(\lambda\) is complex, then \(\vec{v}\) has complex numbers in its components, and more explicit expression can be found.
\end{method}

\faStar{} To reduce the mental load and stress of memorizing seemingly different formulas, you are \emph{strongly encouraged} to compare Method~\ref{method:solutions-for-recurrence-systems} (on page \pageref{eq:solutions-for-recurrence-systems}) with Method~\ref{method:DE-homogeneous-linear-distinct-eigenvalues} and to notice (and remember) that they come from the same \emph{idea}: \hlmain{Express the initial condition as a sum of eigenvectors and the solution is the corresponding sum of \emph{fundamental solutions}.} 

\begin{example}
  Solve the initial-value problem \( \frac{d\vec{x}}{dt} = \begin{bmatrix} 2   & 0  \\ -6 & -1 \end{bmatrix} \vec{x}(t) \) with \(\vec{x}(0) = \begin{bmatrix} 4 \\ 2 \end{bmatrix}\).

  \blanklines{20}
\end{example}
\clearpage
\clearpage

\begin{example}
  Solve 
  \(
    \begin{array}{rcrcr}
      x'_{1}(t) &=& -2 x_{1}(t) &+&  x_{2}(t) \\
      x'_{2}(t) &=&    x_{1}(t) &+& 2x_{2}(t)
    \end{array}
  \) with \(
    \begin{array}{rcr}
      x_{1}(0) &=&  2 \sqrt{5} \\
      x_{2}(0) &=& -2 \sqrt{5} \\
    \end{array}
  \).
  \blanklines{50}
\end{example}
\clearpage

\begin{example}
  Find the general solution for \( \frac{d\vec{x}}{dt} = \begin{bmatrix} 2   & 0  \\ -6 & -1 \end{bmatrix} \vec{x}(t) \).
\end{example}

\subsection{Fundamental solutions for distinct complex eigenvalues} \label{sec:DE-homogeneous-linear-distinct-complex-eigenvalues}

\begin{definition}[fundamental solution for an eigenvector with a \emph{complex} eigenvalue]
  If \(\vec{v}\) is an eigenvector for \(A\) with real eigenvalue \(\lambda\), then the \hlmain{fundamental solution corresponding to \(\vec{v}\)} is 
  \[
    e^{\lambda t} \vec{v}.
  \]

  However, using Euler's identity \(e^{i \theta} = \cos(\theta) + i \sin(\theta)\), we have a further simplification. If \(\lambda = a + bi\) and \(\vec{v} = \vec{A} + \vec{B}i\) where \(a,b,\vec{A},\vec{B}\) only contain real numbers, then
  \[
    e^{\lambda t} \vec{v} = \underbrace{e^{at}(\cos(bt) \vec{A} - \sin(bt) \vec{B})}_{\text{real part}} + i \underbrace{e^{at} \sin(bt) \vec{A} + \cos(bt) \vec{B}}_{\text{imaginary part}}.
  \]
\end{definition}

\begin{example}
  Solve 
\end{example}
\end{document}
