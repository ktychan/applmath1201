%! TeX program = lualatex
\documentclass[../main.tex]{subfiles}
\begin{document} \section{Homogeneous linear systems of DEs with distinct eigenvalues}
  
In this section, we solve linear, first-order systems of differential equations 
\begin{equation} \label{eq:homogeneous-linear-DE}
  \frac{d\vec{x}}{dt} = A \vec{x}(t) \quad\text{with initial condition}\quad \vec{x}(0).
\end{equation}

\begin{method}[solutions of homogeneous {\(2\)-by-\(2\)} systems of DEs with \hlattn{distinct} eigenvalues] \label{method:DE-homogeneous-linear-distinct-eigenvalues}
  Find eigenvectors \(\vec{u}\) with eigenvalue \(\mu\) and \(\vec{v}\) with eigenvalue \(\lambda\).

  We just need \emph{one} eigenvector (with actual numbers) for each eigenvalue. 

  Solve for scalars \hlsupp{\(\alpha, \beta\)} so that \(\vec{x}(0) = \hlsupp{\alpha} \vec{u} + \hlsupp{\beta} \vec{v}\). The solution to Equation~\eqref{eq:homogeneous-linear-DE} is
  \begin{equation} \label{eq:DE-homogeneous-linear-distinct-eigenvalues}
    \vec{x}(t) = \hlsupp{\alpha} \cdot \hlmain{\left( \parbox[c]{1.6in}{\centering fundamental solution associated to \(\vec{u}\)} \right) } + \hlsupp{\beta} \cdot \hlmain{\left( \parbox[c]{1.6in}{\centering fundamental solution associated to \(\vec{v}\)} \right)}.
  \end{equation}

  The \hlmain{fundamental solution} associated to an eigenvector \(\vec{w}\) with eigenvalue \(\theta\) is \hlmain{\(e^{\theta t} \vec{w}\)}.  Therefore, Equation~\ref{eq:DE-homogeneous-linear-distinct-eigenvalues} has a more explicit expression:
  \[
    \vec{x}(t) = \hlsupp{\alpha} \cdot \hlmain{e^{\mu t} \vec{u}} + \hlsupp{\beta} \cdot \hlmain{e^{\lambda t} \vec{v}}.
  \]
\end{method}

You are \emph{strongly encouraged} to compare and contrast Method~\ref{method:solutions-for-recurrence-systems} (on page \pageref{eq:solutions-for-recurrence-systems}) with Method~\ref{method:DE-homogeneous-linear-distinct-eigenvalues} to \emph{reduce} the mental load and \emph{stress} of memorizing seemingly different (they are not) techniques. Notice and remember that the two methods come from the same \emph{idea}: \hlmain{Express the initial condition as a sum of eigenvectors, and then the solution (of the system of DEs) is the corresponding sum of \emph{fundamental solutions}.} 

\begin{example}
  Solve the initial-value problem \( \frac{d\vec{x}}{dt} = \begin{bmatrix} 2   & 0  \\ -6 & -1 \end{bmatrix} \vec{x}(t) \) with \(\vec{x}(0) = \begin{bmatrix} 4 \\ 2 \end{bmatrix}\).

  \blanklines{20}
\end{example}
\clearpage

\begin{example}
  Solve 
  \(
  \begin{array}{rcrcr}
    x'_{1}(t) &=& -2 x_{1}(t) &+&  x_{2}(t) \\
    x'_{2}(t) &=&    x_{1}(t) &+& 2x_{2}(t)
  \end{array}
  \) with \(
  \begin{array}{rcr}
    x_{1}(0) &=&  2 \sqrt{5} \\
    x_{2}(0) &=& -2 \sqrt{5} \\
  \end{array}
  \).
  \blanklines{50}
\end{example}
\clearpage


\faStar{} Method~\ref{method:DE-homogeneous-linear-distinct-eigenvalues} directly gives us specific solutions satisfying some given initial condition.  If the initial condition is not given, then the best we can hope for is a general solution.  \hlmain{To find the general solution, apply Method~\ref{method:DE-homogeneous-linear-distinct-eigenvalues} but simply leave \(\alpha, \beta\) as free variables in Equation~\ref{eq:DE-homogeneous-linear-distinct-eigenvalues}}.  The calculation shortens to finding the eigenvectors for \(A\) and then use Equation~\ref{eq:DE-homogeneous-linear-distinct-eigenvalues}.

Notice that the general solution \emph{does not} have ``\(+C\)'s''.


\begin{example}
  Find the general solution for \( \frac{d\vec{x}}{dt} = \begin{bmatrix} 2   & 0  \\ -6 & -1 \end{bmatrix} \vec{x}(t) \).

  \blanklines{10}
\end{example}


Why does Method~\ref{method:DE-homogeneous-linear-distinct-eigenvalues} work?
\blanklines{30}

\clearpage
Here is an exercise with more involved arithmetic.
\begin{example}
  Solve 
  \(
  \begin{array}{rcrcr}
    x'_{1}(t) &=& -2 x_{1}(t) &+&  x_{2}(t) \\
    x'_{2}(t) &=&    x_{1}(t) &+& 2x_{2}(t)
  \end{array}
  \) with \(
  \begin{array}{rcr}
    x_{1}(0) &=&  2 \sqrt{5} \\
    x_{2}(0) &=& -2 \sqrt{5} \\
  \end{array}
  \).
  \blanklines{50}
\end{example}

If the matrix \(A\) in \(\frac{d\vec{x}}{dt} = A \vec{x}(t)\) has complex eigenvalues, then the fundamental solutions have explicit expression. 

\begin{definition}[fundamental solution for an eigenvector with \emph{complex} eigenvalues]
  Euler's identity \(e^{i \theta} = \cos(\theta) + i \sin(\theta)\) gives us more explicit (but longer) expressions for fundamental solutions. 

  Suppose the matrix \(A\) in \(\frac{d\vec{x}}{dt} = A \vec{x}(t)\) has complex eigenvalue \(\lambda = a + bi\) with an eigenvector \(\vec{v} = \vec{p} + \vec{q}i\) where \(a,b,\vec{p},\vec{q}\) are all real. The \hlmain{fundamental solution for \(\vec{v}\)} is
  \[
    e^{\lambda t} \vec{v} = \underbrace{e^{at} \bigg( \cos(bt) \vec{p} - \sin(bt) \vec{q} \bigg)}_{\text{real part}} +  \underbrace{e^{at} \bigg( \sin(bt) \vec{q} + \cos(bt) \vec{q}\bigg)}_{\text{imaginary part}} i.
  \]

  Remember the \(i\) at the end.
\end{definition}

\begin{example}
  Solve \(\frac{d\vec{x}}{dt} = \begin{bmatrix} 0 & 1 \\ -1 & 0 \end{bmatrix}\vec{x}(t)\) with initial condition \(\vec{x}(0) = \begin{bmatrix} 1 \\ 1 \end{bmatrix}\).
\end{example}
\end{document}
