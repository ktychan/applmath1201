%! TeX program = lualatex
\documentclass[../main.tex]{subfiles}
\begin{document} \section{Probability}

% Introduce random variables as data recording devices.

Probability gives us a computable language to describe anything with natural uncertainty, such as the effectiveness of a treatment.

\begin{definition}[sample spaces and outcomes]
  A \hlmain{sample space} is a set, typically denoted as \(\Omega\).  An element of \(\Omega\) is called an \hlmain{outcome}.  

  In other words, the sample space consists of all possible outcomes (of an experiment).
\end{definition}

\begin{example} \label{ex:probability-die-1}
  Suppose we perform an experiment by rolling a six-sided die \emph{once}.  The outcomes (plural!) of this experiment are \underline{\hspace{2in}}.  The sample space (singular!) of this experiment is \underline{\hspace{2in}}.
\end{example}

Sometimes, we don't care about a particular outcome but a bunch outcomes together. Instead of saying a bunch of outcomes, we use the fancy term \emph{events}.

\begin{definition}[events]
  An \hlmain{event} in a sample space \(\Omega\) is a subset of \(\Omega\).  Intuitive, an event is a bunch of outcomes, and an event contains outcomes.

  \faExclamationTriangle{} It \emph{does not make sense} to think about an event without its enclosing sample space.
\end{definition}
In probability, the word \emph{event} should be understood in the same sense as ``\emph{in the event of...}.'' Saying that \(E\) is an event \emph{does not} mean such an event has already happened.

\begin{example} \label{ex:probability-die-2}
  Continue from Example~\ref{ex:probability-die-1}.  Describe the event that a die-roll is even.
  \blanklines{5}
\end{example}

Events and their sample space \emph{together} give us the language of probability.
\begin{definition}[probability]
  The \hlmain{probability of an event} \(E\) in a sample space \(\Omega\) is defined to be
  \begin{equation} \label{eq:probability}
    \mathbb{P}(E) = \frac{\text{number of outcomes in \(E\)}}{\text{total number of outcomes in \(\Omega\)}}.
  \end{equation}
\end{definition}
A probability is a number \underline{\hspace{3in}}. To calculate a probability from the first principle is to identify the sample space, the event and \emph{then} use Equation~\eqref{eq:probability}.
\begin{example}
  Find the probability of getting an even number after rolling a six-sided die \emph{once}.
  \blanklines{6}
\end{example}
\clearpage

\begin{example}
  We roll two four-sided dice \emph{once}.  Assuming the faces are labelled \(1,2,3,4\), what is the probability that the sum is \(6\)?

  \blanklines{15}
\end{example}


\faStar{} Equation~\eqref{eq:probability} tells us that calculating a probability from the first principle is essentially a counting problem. Less obvious is that counting can get rather challenging. In \thecoursesubject~\thecoursenumb, we will try to avoid the first principle approach to probability whenever possible and use relations among events to calculate probabilities of events by relating them to probabilities of known events.

We recall set operations. Given two events \(E,F\) in a sample space \(\Omega\), define
\[
  E \cap F = \{\text{ outcomes in \(E\) and in \(F\) } \} \quad\text{and}\quad
  E \cup F = \{\text{ outcomes in \(E\) or in \(F\) } \}.
\]
\begin{center}
  \includegraphics{../standalones/build/diagram-venn-2sets}
  \hspace{3em}
  \includegraphics{../standalones/build/diagram-venn-2sets}
\end{center}
{\footnotesize A little memory trick: The \(\cap\) symbol looks sort of like the \(A\) in AND.}

\faStar{} We have a relation
\begin{align} \label{eq:probability-cup-cap}
  \mathbb{P}( E \cup F ) = \hspace{4in}
\end{align}

Denote \(E^{c} = \{ \text{ outcomes in \(\Omega\) but not in \(E\) } \}\). Equation~\eqref{eq:probability-cup-cap} becomes
\begin{align} \label{eq:probability-comp}
  \mathbb{P}( E^{c} ) = \hspace{4in}
\end{align}
\blanklines{5}
\clearpage

It is often natural to assign every outcome in a sample space to exactly one event. The probability lingo is \emph{mutually exclusive} and \emph{mutually exclusive and exhaustive}.

\begin{definition}[mutually exclusive and exhaustive]
  Two events \(E, F\) in a sample space \(\Omega\) are \hlmain{mutually exclusive} if \(E \cap F\) is empty. 

  A bunch events \(E_{1}, \ldots, E_{n}\) in a sample space \(\Omega\) are \hlmain{mutually exclusive and exhaustive} if \hlattn{every} outcome in \(\Omega\) belongs to \hlattn{exactly one} of these events, and no outcome is left behind. The precise relations are
  \[
    \Omega = E_{1} \cup \cdots \cup E_{n} \quad\text{ and }\quad E_{i} \cap E_{j} = \emptyset \text{ for all } i \ne j. 
  \]
\end{definition}
\blanklines{8}

\begin{example}
  Assume \(E_{1}, E_{2}\) are events in a sample space \(\Omega\).  Which of the followings \emph{together} confirms that \(E_{1}, E_{2}\) are mutually exclusive and exhaustive. 
  \begin{enumerate}[label=(\alph*)]
    \item \(\mathbb{P}(E_{1} \cap E_{2}) = 0\).
    \item \(\mathbb{P}(E_{1} \cap E_{2}) = \mathbb{P}(E_{1}) + \mathbb{P}(E_{2})\).
    \item \(\mathbb{P}(E_{1} \cup E_{2}) = \mathbb{P}(E_{1}) + \mathbb{P}(E_{2})\).
    \item \(\mathbb{P}(E_{1} \cup E_{2}) = 0\).
    \item \(\mathbb{P}(E_{1} \cup E_{2}) = 1\).
  \end{enumerate}
\end{example}
\blanklines{5}

\faExclamationTriangle{} When there are three or more events, checking
\[
  \mathbb{P}(E_{1} \cup \cdots \cup E_{n}) = 1 \quad\text{and}\quad \mathbb{P}(E_{1} \cap \cdots \cap E_{n}) = 0
\]
is NOT ENOUGH to guarantee \(E_{1}, \ldots, E_{n}\) are mutually exclusive and exhaustive.

\end{document}
