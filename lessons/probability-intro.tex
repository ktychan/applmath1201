%! TeX program = lualatex
\documentclass[../main.tex]{subfiles}
\begin{document} \section{Probability}

\begin{objective}
  \begin{enumerate}
    \item the \hlmain{sample} space of an experiment,
    \item an \hlmain{event} in the sample space,
    \item set complement, union and intersection,
    \item apply the definition of probability to calculate the probability of an event occurring in an en experiment,
  \end{enumerate}
\end{objective}

% Introduce random variables as data recording devices.

Consider a certain population. Let \(X_{1}\) record heights and \(X_{2}\) weights. More precisely, if \(i\) is someone in the population, then \(X_{1}(i)\) is the height of \(i\), and \(X_{2}(i)\) is the weight of \(i\).

\begin{definition}[sample space]
  A \hlmain{sample space} is a set, typically denoted as \(\Omega\).
\end{definition}

An \hlmain{event} is a subset of the event space, typically denoted by \(E \subseteq \Omega\).

The probability of an event \(E\) in a sample space \(\Omega\) is defined to be
\[
  \mathbb{P}(E) = \frac{\text{number of outcomes in \(E\)}}{\text{total number of outcomes in \(\Omega\)}}
\]

\begin{example}
  Let \(\Omega\) be the sample space of everyone in our lecture hall.  
\end{example}

Venn diagrams.
\end{document}
