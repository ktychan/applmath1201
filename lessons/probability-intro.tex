%! TeX program = lualatex
\documentclass[../main.tex]{subfiles}
\begin{document} \section{Probability}

Probability gives us a computable language to describe anything with natural uncertainty, such as the effectiveness of a treatment.

\begin{definition}[sample spaces, outcomes and events]
  A \hlmain{sample space} is a set, typically denoted as \(\Omega\).  An element of \(\Omega\) is called an \hlmain{outcome}.   In other words, the sample space consists of all possible outcomes (of an experiment).

  An \hlmain{event} in a sample space \(\Omega\) is a subset of \(\Omega\).  Intuitive, an event is a bunch of outcomes, and an event contains outcomes.
\end{definition}
In probability, the word \emph{event} should be understood as ``\emph{in the event of...}.'' Saying that \(E\) is an event \emph{does not} mean such an event has already happened.

\blanklines{5}

\begin{example} \label{ex:probability-die-1}
  Suppose we perform an experiment by rolling a six-sided die \emph{once}.  Describe the outcomes, the sample space and the event of rolling an even number. 

  \blanklines{10}
\end{example}

Events and their sample space \emph{together} give us the language of probability.
\begin{definition}[probability]
  The \hlmain{probability of an event} \(E\) in a sample space \(\Omega\) is defined to be
  \begin{equation} \label{eq:probability}
    \mathbb{P}(E) = \frac{\text{number of outcomes in \(E\)}}{\text{total number of outcomes in \(\Omega\)}}.
  \end{equation}
\end{definition}
A probability is a number \underline{\hspace{3in}}. To calculate a probability from the first principle is to identify the sample space, the event and \emph{then} use Equation~\eqref{eq:probability}.
\begin{example}
  Find the probability of getting an even number after rolling a six-sided die \emph{once}.
  \blanklines{6}
\end{example}
\clearpage

\begin{example}
  We flip a coin twice.  What is the probability that we get at least one tail?
  \blanklines{20}
\end{example}

\begin{example}[human genetics]
  Consider a simple genetic trait controlled by a single gene \(A\) which can be either dominant (denoted by \(A+\)) or recessive (denoted by \(A-\)).

  Humans carry two copies of most genes, one from each biological parent. Suppose a child inherits a trait from their parents who each carries one dominant and one recessive gene. What is the probability that the child carries
  \begin{enumerate}
    \item one dominant and one recessive gene?
    \item at least one recessive gene?
  \end{enumerate}

  \blanklines{20}
\end{example}
\clearpage

\begin{example}
  We roll two four-sided dice \emph{once} one blue and one red.  Assuming the faces are labelled \(1,2,3,4\), what is the probability that the sum is \(6\)?

  \blanklines{15}
\end{example}

\begin{example}
  Suppose we have two bags of marbles.  The first bag has a white and a black marble.  The second bag has a red, a blue and a green marble.

  Consider an experiment where we draw one marble from each bag.  Determine the sample space and find the probability of drawing a white and a green marble. 

  \blanklines{15}
\end{example}

\begin{example}
  Allele are variants of genes that determine traits.  A gene that determine eye colour might have an allele for brown eyes and an allele for blue eyes.  

  Suppose for a particular trait, an organism inherits \(0,1,2\) allele from its mother and \(0\) or \(1\) alleles from its father, what is the probability that the organism inherits \(2\) alleles?

  \blanklines{10}
\end{example}
\clearpage

Equation~\eqref{eq:probability} tells us that calculating a probability from the first principle is essentially a counting problem. The language of sets comes in handy. % Less obvious is that counting can get rather challenging. In \thecoursesubject~\thecoursenumb, we will try to avoid the first principle approach to probability whenever possible and use relations among events to calculate probabilities of events by relating them to probabilities of known events.
Given two events \(E,F\) in a sample space \(\Omega\), define
\[
  E \cap F = \{\text{ outcomes in \(E\) and in \(F\) } \} \quad\text{and}\quad
  E \cup F = \{\text{ outcomes in \(E\) or in \(F\) } \}.
\]
\begin{center}
  \includegraphics{../standalones/build/diagram-venn-2sets}
  \hspace{3em}
  \includegraphics{../standalones/build/diagram-venn-2sets}
\end{center}
{\footnotesize A little memory trick: The \(\cap\) symbol looks sort of like the \(A\) in AND.}

We have a relation, called the \emph{inclusion-exclusion principle}, 
\begin{align} \label{eq:probability-cup-cap}
  \mathbb{P}( E \cup F ) = \hspace{4in}
\end{align}

Once we define an event called \(E\), then we can talk about outcomes \emph{not} in \(E\).  
\begin{center}
  \includegraphics{../standalones/build/diagram-venn-complement}
\end{center}

Denote \(E^{c} = \{ \text{ outcomes in \(\Omega\) but not in \(E\) } \}\).  We have an associated formula
\begin{align} \label{eq:probability-comp}
  \mathbb{P}( E^{c} ) = \hspace{4in}
\end{align}

\faStar{} It is important to \emph{remember} and know how to \emph{use} Equations~\ref{eq:probability-cup-cap}~and~\ref{eq:probability-comp}.

Venn diagrams are very useful for seeing relations among sets and calculating probabilities. 
\begin{example}
  Suppose \(A = \{1\}, B = \{1,2,4\}, C = \{1,3,5\}\) are events in a sample space \(\Omega = \{1,2,3,4,5\}\). Use a Venn diagram to find the elements of \( (A \cup B)^{c} \cap C^{c}\) and calculate \(\mathbb{P}( (A \cup B)^{c} \cap C^{c} )\).
  \blanklines{10}
\end{example}

\clearpage

\begin{example}
  Suppose the sample space is \(\Omega = \{1,2,\ldots,8\}\). Let \(A = \{1,2,8\}\), \(B = \{2,5,8\}\) and \(C = \{3,4,5,6\}\).  Find the probability \(\mathbb{P}( C^{c} \cup (A \cap B) )\).
  
  \blanklines{15}
\end{example}

\begin{example}
  Suppose the sample space is \(\Omega = \{1,2,\ldots,10\}\).  Let \(E = \{1,3,5\}\) and \(F = \{2,7,3,4\}\). Find the probability \(\mathbb{P}( (E \cap F) \cup (E \cap F^{c})) \).
  \blanklines{15}
\end{example}

\begin{example}
  Suppose the sample space is \(\Omega = \{1,2,3,4,5\}\).  Let \(E = \{2,5\}\), \(F = \{1,2,3\}\) and \(G = \{2,4\}\). Find the probability \(\mathbb{P}( (E \cap G^{c}) \cap \{ F^{c} \cup E^{c} \}) \).
  \blanklines{13}
\end{example}
\clearpage

\begin{example}
  Suppose \(\mathbb{P}(E) = 0.6\), \(\mathbb{P}(F) = 0.45\) and \(\mathbb{P}(E \cap F) = 0.4\). Calculate \(\mathbb{P}( (E \cup F)^{c})\).
  \blanklines{10}
\end{example}

\begin{example}
  Consider events \(E\) and \(F\) inside some sample space. Given \(\mathbb{P}(E) = 0.5\), \(\mathbb{P}(F^{c}) = 0.8\), \(\mathbb{P}(E \cap F) = 0.05\). Find the probability of \(\mathbb{P}(E \cup F^{c})\).
  \blanklines{10}
\end{example}

\begin{example}
  Toss a coin and roll a four-sided die labelled \(1,2,3,4\).  Find the probability of getting a tail and an even number.

  \blanklines{10}
\end{example}

\begin{example}
  You randomly draw two balls from \(\{1,2,3,4,5\}\).  Repeats are not allowed, e.g., you cannot draw \(3\) twice.  What is the probability that the sum is at least \(3\)?

  \blanklines{13}
\end{example}


\clearpage

% We can calculate probability of a sequence of outcomes using a tree (branching) diagram.  It is easier to describe this though an example. 
%
% \begin{example}
%   Let's roll a four-sided die and then toss a coin.  Find the probability of getting a \(3\) and a tail.
%
%   \blanklines{10}
% \end{example}
%
% \begin{example}
%   Toss a coin three times in a row.  Find the probability of getting two heads and one tail in any order. 
%
%   \blanklines{10}
% \end{example}
%
% \begin{example}
%   Toss a coin, roll a four-sided die, then toss a coin again. What is the probability of getting \(2\) tails and a \(1\) in any order?
%
%   \blanklines{10}
% \end{example}
% \clearpage
%
% Tree (branching) diagrams nicely capture of the probabilities of gene mutations. 
% \begin{example}
%   The sex of a bee is determined by the number of chromosomes an individual possesses. Diploids are cells that carry two sets of genes, one from each parent. In contrast, haploids only carry one set of chromosome from their mother.
%
%   Female bees are diploids.   Male bees are haploids.
%
%   What is the probability that an allele currently carried by a male bee descended from a male exactly \(2\) generations ago?
%
%   \blanklines{40}
% \end{example}
% \clearpage

It is often natural to assign every outcome in a sample space to exactly one event. The probability lingo is \hlmain{mutually exclusive} and \hlmain{exhaustive}. In real-life applications, \emph{mutually exclusive and exhaustive} events capture groupings by non-overlapping characteristics within the sample space, e.g., grouping of patients by age group.

\begin{definition}[mutually exclusive and exhaustive]
  Two events \(E, F\) in a sample space \(\Omega\) are \hlmain{mutually exclusive} if \(E \cap F\) is empty. 

  Events \(E_{1}, \ldots, E_{n}\) in a sample space \(\Omega\) are \hlmain{mutually exclusive and exhaustive} if \hlattn{every} outcome in \(\Omega\) belongs to \hlattn{exactly one} of these events, no outcome is left behind and no outcome outside of the sample space is included. The precise relations are
  \[
    \Omega = E_{1} \cup \cdots \cup E_{n} \quad\text{ and }\quad E_{i} \cap E_{j} = \emptyset \text{ for all } i \ne j. 
  \]
\end{definition}
\blanklines{15}


% \begin{example}
%   The next Ontario election is on February 27, 2025 (\url{https://www.elections.on.ca}). Let \(\Omega\) be Ontario voters. Which of the following describe mutually exclusive and exhaustive events for \(\Omega\)?
%
%   \begin{enumerate}[label=(\alph*)]
%     \item \(\{ \text{ under 21 }\}, \{ \text{ university students } \}, \{ \text{ everybody else } \}\)
%     \item \(\{ \text{ born in ON }\}, \{ \text{ born in QC } \}, \{ \text{ born elsewhere in Canada }\}, \{ \text{ born outside of Canada }\} \)
%     \item \(\{ \text{ no high school diploma } \}, \{ \text{ has high school diploma } \}, \{ \text{ has a university degree } \}\) 
%     \item \(\{ \text{ MB residents } \}, \{ \text{ ON residents } \}, \{ \text{ QC residents }\}, \{ \text{ everybody else }\}\) 
%   \end{enumerate}
% \end{example}

\begin{example}
  Let \(\Omega = \{1,2,3,4,5,6\}\) be the sample space. Which of the following events are mutually exclusive?

  \begin{multicols}{2}
    \begin{enumerate}[label=(\alph*)]
      \item \(E_{1} = \{2,5\}\)
      \item \(E_{2} = \{3,2\}\)
      \item \(E_{3} = \{6,3\}\)
      \item \(E_{4} = \{2,3,5,6\}\)
      \item \(E_{5} = \{1\}\)
    \end{enumerate}
  \end{multicols}
  \blanklines{2}
\end{example}


\begin{example}
  Let \(\Omega = \{1,2,,\ldots,10\}\) be the sample space. Which events together are mutually exclusive and exhaustive?

  \begin{multicols}{2}
    \begin{enumerate}[label=(\alph*)]
      \item \(E_{1} = \{1,3,5\}\)
      \item \(E_{2} = \{8,9\}\)
      \item \(E_{3} = \{3,7\}\)
      \item \(E_{4} = \{4,10\}\)
      \item \(E_{5} = \{2,6,7\}\)
    \end{enumerate}
  \end{multicols}
  \blanklines{2}
\end{example}
\clearpage

\begin{example}
  Assume \(E_{1}, E_{2}\) are events in a sample space \(\Omega\).  Which of the followings \emph{together} imply that \(E_{1}, E_{2}\) are mutually exclusive and exhaustive. 
  \begin{multicols}{2}
    \begin{enumerate}[label=(\alph*)]
      \item \(\mathbb{P}(E_{1} \cap E_{2}) = 0\).
      \item \(\mathbb{P}(E_{1} \cap E_{2}) = \mathbb{P}(E_{1}) + \mathbb{P}(E_{2})\).
      \item \(\mathbb{P}(E_{1} \cup E_{2}) = \mathbb{P}(E_{1}) + \mathbb{P}(E_{2})\).
      \item \(\mathbb{P}(E_{1} \cup E_{2}) = 0\).
      \item \(\mathbb{P}(E_{1} \cup E_{2}) = 1\).
    \end{enumerate}
  \end{multicols}

  \blanklines{10}
\end{example}


\begin{example}[a common mistake] \label{ex:mutually-exclusive-three-or-more}
  Let \(\Omega = \{1,\ldots,10\}\) be a sample space.  An intern observes the following. Events \(E_{1}, E_{2}, E_{2}\) have no outcomes in common and every outcome in the sample space is in at least one of the events. Express the intern's observation using set operations.
  \blanklines{5}

  Describe three sets \(E_{1}, E_{2}, E_{3}\) in the given sample space matching the intern's observation but are not mutually exclusive and exhaustive.  In other words, prove the intern wrong.
  \blanklines{5}

  \faExclamationTriangle{} Example~\ref{ex:mutually-exclusive-three-or-more} serves as a warning. When there are \hlwarn{three or more} events, being able to confirm
  \[
    \mathbb{P}(E_{1} \cup \cdots \cup E_{n}) = 1 \quad\text{and}\quad \mathbb{P}(E_{1} \cap \cdots \cap E_{n}) = 0
  \]
  is \hlwarn{not enough} to guarantee \(E_{1}, \ldots, E_{n}\) are mutually exclusive and exhaustive. To guarantee that three or more events are mutually exclusive, we must be able to confirm the following conditions
  \[
    \mathbb{P}(E_{1} \cup \cdots \cup E_{n}) = 1 \quad\text{and}\quad \mathbb{P}(E_{i} \cap E_{j}) = 0 \text{ for all } i \ne j.
  \]
\end{example}
\end{document}
