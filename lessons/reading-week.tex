%! TeX program = lualatex
\documentclass[../main.tex]{subfiles}
\begin{document} \section{Reading week}

Enjoy some down time! 

Here are some (but not comprehensive) review exercises.  

\begin{example}
  Consider events \(A,B\) in a common sample space such that \(\mathbb{P}(A) = 0.5\), \(\mathbb{P}(B) = 0.7\) and \(\mathbb{P}(A \cap B) = 0.35\).  

  \begin{enumerate}
    \item Are \(A,B\) independent?
    \item Are \(A,B\) mutually exclusive? 
  \end{enumerate}

  \blanklines{5}
\end{example}

\begin{example}
  Consider events \(A,B\) in a common sample space such that \(\mathbb{P}(A \mid B) = 0.3\) and \(\mathbb{P}(A) = 0.2\). Are \(A,B\) independent?

  \blanklines{5}
\end{example}

\begin{example}
  Consider independent events \(E,F\) in a common sample space such that \(\mathbb{P}(E) = 0.25\). What is \(\mathbb{P}(E \mid F)\)?

  \blanklines{5}
\end{example}

\begin{example}
  If \(E_{1},E_{2},E_{3}\) are mutually exclusive and exhaustive events in a sample space, are they independent?
  \blanklines{5}
\end{example}

\begin{example}
  Suppose \(B_{1},B_{2},B_{3},B_{4}\) are mutually exclusive events in a sample space. We know that \(\mathbb{P}(E \cap B_{1}) = 0.1\), \(\mathbb{P}(E \cap B_{2}) = 0.2\), \(E\) and \(B_{3}\) are mutually exclusive, and \(E\) and \(B_{4}\) have nothing in common. Calculate \(\mathbb{P}(E)\).

  \blanklines{5}
\end{example}
\clearpage

\begin{example}
  A local health agency tested \(2000\) birds for a strain of avian flu.  The specificity of the test is \(20\%\) and sensitivity \(95\%\).  The number of birds later confirmed carrying the avian flu is \(500\). The other birds did not have the avian flue.

  \begin{enumerate}[wide]
    \item Should you trust a positive result of this test?
      \blanklines{5}
    \item Should you trust a negative result of this test?
      \blanklines{5}
    \item How many birds tested positive?
      \blanklines{10}
    \item How many birds tested negative?
      \blanklines{10}
  \end{enumerate}
\end{example}
\clearpage

\begin{example}
  A clinic tested \(100\) people for migraine.  The probability of type-I error is \(10\%\) and type-II error is \(90\%\).  The number of people later confirmed having migraine is \(80\). The other people did not have migraine.

  \begin{enumerate}[wide]
    \item Do you feel confident using the test to rule out migraine? How about to confirm migraine?
      \blanklines{10}
    \item What is the sensitivity of this test? What is its specificity?
      \blanklines{10}
    \item How many birds tested positive? How many tested negative?
      \blanklines{10}
  \end{enumerate}
\end{example}
\end{document}

