%! TeX program = lualatex
\documentclass[../main.tex]{subfiles}
\begin{document} \section{Introduction to continuous-time models and differential equations}

\hlmain{Continuous-time models} are models in which the time flows continuously as experienced by human. In mathematical terms, the time variable \(t\) takes values over an interval. 

\hlmain{Differential equations} are equations for describing changes over time and are often used to describe continuous-time models.

We introduce these two ideas together through Example~\ref{ex:diff-eq-intro}.

\begin{example} \label{ex:diff-eq-intro}
  Let \(N\) measure the size of a population (e.g., cells, animals, any \emph{group} of organisms that live through some kind of life-cycles). Assume time \(t\) is measured in \emph{days}. At any given time \(t\), the \emph{per capita} birth and death rates are constants \(B\) and \(D\).  

  \emph{Per capita} means per unit of population, e.g., per person, per cell, per unicorn, etc.
  \blanklines{15}
  % explain beta as 1 |-> self + 0.5 self.

  Express the rate of change of the population \hlmain{at time \(t\)} as an equation.  

  This is an example of a differential equation.
  \blanklines{10}

  What do we \emph{gain} by studying differential equations?
  \blanklines{5}
\end{example}

\clearpage
Let's sharpen the concept of differential equations by writing down a precise definition.
\begin{definition}[differential equation]
  Let \(u\) be a differentiable function of \(t\). A \hlmain{differential equation} is an equation that relates \(u\) to one or more derivatives. 

  In other words, a differential equation is an equation relating an independent variable, a dependent variable and derivatives of \underline{\hspace{3in}}.

  \faExclamationTriangle{} Differential equations are used to describe continuous-time models.
\end{definition}

\faStar{} Differential equations are continuous-time analog of recurrence equations (not the whole recursion).  

\begin{example}
  Here are some examples of differential equations.  Explain each differential equation in words.

  \begin{enumerate}[wide]
    \item A differential equation can be written using function notations. Identify the dependent and independent variables in the equation.
      \begin{equation}
        N'(t) = r N(t) \left(1-\frac{N(t)}{K}\right)
      \end{equation}
      \blanklines{3}

    \item A differential equation can be written without function notations.
      \begin{equation}
        u' = u (1-u/K)
      \end{equation}
      \blanklines{5}

    \item A differential equation can involve any algebraic operations.
      \begin{equation}
        \frac{dy}{dx} + xy = 0
      \end{equation}
      \blanklines{3}

    \item A differential equation can be organized in anyway and have high-order derivatives.
      \begin{equation}
        v(t) = v'(t) + \frac{v''(t)}{t} + \sqrt{t-1}
      \end{equation}
      \blanklines{3}
  \end{enumerate}
\end{example}
\end{document}

