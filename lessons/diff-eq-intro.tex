%! TeX program = lualatex
\documentclass[../main.tex]{subfiles}
\begin{document} \section{Introduction to differential equations}

So far, we can apply elementary algebraic operations \(+, -, \times, \div, \circ\) to elementary functions to develop models. Are these tools all we need to capture sufficient details of nature? Let's find out.

\begin{example}[The good ol' population model]
  In prerequisite courses, we learned that the relation between a population \(P\) and time \(t\) can be modelled by
  \begin{equation} \label{eq:exponential-growth}
    P(t) = P_{0} \, e^{rt}.
  \end{equation}
  \blanklines{5}

  \faComment{} What details related to population growth and decline are ignored by this model?
  \blanklines{15}
\end{example}

\clearpage
Let's develop a more realistic population model.

\begin{mdframed}[style=simple-compact] \label{def:carrying-capacity}
  \textbf{Definition} (carrying capacity). The environment in which a population \(P\) lives imposes a \hlmain{carrying capacity} \(K\) to the population. The carrying capacity is a constant with respect to the population \(P\) and time \(t\).  
\end{mdframed}

Empirical studies suggest that \(K\) interacts with \(P\) as follows.
\begin{enumerate}[label=(L\arabic*)]
  \item If \(P > K\), then the population shrinks due to a lack of resource.
  \item If \(P = K\), then the population does not change.
  \item If \(P < K\), then the population grows due to the availability of resources.  There are two subcases.
    \begin{enumerate}
      \item If \(P\) is much smaller than \(K\), then \hlmain{the population growth at time \(t\)} is \hlattn{proportional} to \hlmain{the size of the population at time \(t\)} by a positive factor \(r\).
      \item If \(P\) is very close to \(K\), then \hlmain{the population growth at time \(t\)} is \hlattn{proportional} to \hlmain{the size of the population at time \(t\)} by a positive factor very close to \(0\).
    \end{enumerate}
\end{enumerate}

Describe (L1), (L2), (L3a) and (L3b) individually using calculus. Recall that the derivative is the mathematical way to describe a rate of change. 
\blanklines{30}

\begin{mdframed}[style=simple-compact] \label{def:logistic-eq}
  \textbf{Definition} (logistic equation)
  Suppose \(K\) and \(r\) are positive constants. The \hlmain{logistic equation} is
  \begin{equation} \label{eq:logistic}
    P' = r P \cdot \left( 1 - \frac{P}{K} \right).
  \end{equation}
\end{mdframed}
\blanklines{2}

\begin{example}
  Verify that the logistic equation describes the interaction between \(K\) and \(P\). 

  \begin{enumerate}[wide, label=(L\arabic*)]
    \item Show that if \(P > K\), then \(P' < 0\).
      \blanklines{5}
    \item Show that if \(P = K\), then \(P' = 0\).
      \blanklines{5}
    \item For the two sub-cases, use words to explain your reasoning whenever convenient. 

      Hint: In each case, is \(1 - P/K\) closer to \(0\) or \(1\)?
      \begin{enumerate}[wide]
        \item Argue that if \(P < K\) and \(P\) is much smaller than \(K\), then \(P'(t)\) is proportional to \(P(t)\) by a positive factor of \(r\). 
          \blanklines{8}
        \item Argue that if \(P < K\) and \(P\) is very close to \(K\), then \(P'(t)\) is proportional to \(P(t)\) by a positive factor close to \(0\).
          \blanklines{8}
      \end{enumerate}
  \end{enumerate}
\end{example}
\clearpage

The logistic equation is just the tip of a iceberg called differential equations.

\begin{mdframed}[style=simple-compact] \label{def:diff-eq}
  \textbf{Definition} (differential equations). Let \(u\) be a function of \(t\). An equation involving \(t,u, u', u^{(2)}, u^{(3)}, u^{(4)}, \ldots\) is called a differential equation.
\end{mdframed}

Differential equations are very useful modelling tools in biology (and science in general) because they provide a convenient language to describe natural phenomena.  The above exercise with the logistic equations demonstrates a typical modelling workflow. Differential equations are used to describe observations from empirical studies. One then tries to use the differential equation (find a solution, approximate, argue qualitatively, etc.) to support their scientific hypothesis.

\faExclamationTriangle{} In a differential equation, the \hlmain{dependent} variable \(u\) is the \hlmain{unknown}. The goal is to understand the dependent variable \(u\). 

\faStar{} \textbf{Learning objectives}. By the end of this week, we will be able to
\begin{enumerate}
  \item verify whether a function is a solution to a differential equation, 
  \item classify differential equations by order and how dependent variables appear (linear vs non-liner),
  \item determine whether a differential equation is autonomous or not, 
  \item find general solutions to linear, first-order differential equations, and
  \item find solutions to initial value problems.
\end{enumerate}

\begin{example}
  Consider the following equations. Are they differential equations?

  \begin{enumerate}
    \item \(P' = r (1 - P/K) P\) where \(K, r\) are positive constants and \(P\) is a function of \(t\).
    \item \(u' - u/t - 1 = 0\) where \(u\) is a function of \(t\).
    \item \(y y'' - u' = \sin(t)\) where \(y\) is a function of \(x\).
    \item \(g(y'') = xy^{2}\) where \(y\) is a function of \(x\) and \(g\) is some function.
  \end{enumerate}
\end{example}
\end{document}
