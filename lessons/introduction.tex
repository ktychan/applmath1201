%! TeX program = lualatex
\documentclass[../main.tex]{subfiles}
\begin{document} \section{Introduction to mathematical models}
As scientists, we observe nature and try to create reasonable predictions of the physical world. Scientific theory describes the real world using models. Here are some familiar models.

\begin{enumerate}
  \item The solar system \url{https://eyes.nasa.gov/apps/solar-system/}
  \item Newton's \(F = ma\).
  \item The law of supply and demand in economics.
  \item \(PV = nRT\).
\end{enumerate}

Let's \hlmain{learn to ask basic questions} about models (instead of taking them for granted because someone in charge of our grades says so).

\faComment{} Answer the following questions using \hlmain{think-pair-share}. \hlmain{Think} about them for 2 minutes on your own and write them down in the space below. \hlmain{Discuss} your thoughts with your neighbours. Lastly, \hlmain{share} it with the class. 

\bigskip
\begin{enumerate}[wide]
  \item Do they make sense? 
    \blanklines{5}

  \item Are they accurate in every possible way? 
    \blanklines{5}

  \item Do they provide knowledge and insights? 
    \blanklines{5}

  \item Are they relatively easy to work with? 
    \blanklines{5}
\end{enumerate}

\clearpage

\begin{definition}[mathematical models]
  A \hlmain{mathematical model} is one equation or a system of equations, typically with more than one variable, that describes the world around us.
\end{definition}

What do we learn in \thecoursesubject~\thecoursenumb?

\begin{enumerate}
  \item \textbf{Describe}. Formulate equations to describe scientific phenomena.

  \item \textbf{Tools}. A collection of mathematical objects typically used in modelling.
    \begin{itemize}
      \item Recursion.
      \item Differential equations.
      \item Probability and random variables.
      \item Vectors, functions of vectors and vector fields.
      \item Systems of linear differential equations.
    \end{itemize}

  \item \textbf{Techniques}. Use mathematics (mostly calculus) to understand properties of models.

  \item \textbf{Research}. Use all of the above to \emph{shape} and \emph{defend} our scientific opinions (e.g., models) and make predictions and comparisons.
\end{enumerate}

\end{document}
