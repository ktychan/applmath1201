%! TeX program = lualatex
\documentclass[../main.tex]{subfiles}
\begin{document} \section{Introduction to mathematical models}
As scientists, we observe nature and try to create reasonable predictions of the physical world. Scientific theory describes the real world using models. Here are some familiar ones.

\begin{enumerate}
  \item The solar system \url{https://eyes.nasa.gov/apps/solar-system/}
  \item Newton's second law \(F = ma\).
  \item The law of supply and demand in economics.
  \item The ideal gas law \(PV = nRT\).
  \item The exponential (population) growth \(P = P_{0} e^{rt}\).
\end{enumerate}

Let's \hlmain{learn to ask basic questions} about models (instead of taking them for granted because someone in charge of our grades says so).

\faComment{} Pick two or more models from above and answer the following questions using \hlmain{think-pair-share}. \hlmain{Think} about them for 2 minutes on your own and articulate your thoughts in the space below. \hlmain{Pair} up with one or more neighbours. \hlmain{Share} your thoughts.

\bigskip
\begin{enumerate}[wide]
  \item Do they match our everyday experience? Do they make sense?
    \blanklines{5}

  \item Are they accurate in every possible way?  Do they have caveats and limitations?
    \blanklines{5}

  \item Are they built on certain assumptions? What happens when these assumptions break?
    \blanklines{5}
    % The law of supply and demand assume people are rational actors of the economy.  This is not always true.

  \item Are they relatively easy to work with? 
    \blanklines{5}
\end{enumerate}

\clearpage

\begin{definition}[mathematical models] \label{def:models}
  A \hlmain{mathematical model} is one equation or a system of equations, typically with more than one variable, that describes the world around us.
\end{definition}

What do we learn in \thiscourse{}?

\begin{enumerate}
  \item \textbf{Ideas}. How to formulate equations to describe scientific phenomena.

  \item \textbf{Models}. Some fundamental discrete-time and continuous-time models.

  \item \textbf{Tools}. A collection of mathematical objects typically used in modelling.
    \begin{itemize}
      \item Recursion equations for discrete-time models.
      \item Differential equations for continuous-time models.
      \item Probability and random variables.
      \item Vectors, functions of vectors and vector fields.
      \item Systems of linear differential equations.
    \end{itemize}

  \item \textbf{Techniques}. Use mathematics (mostly calculus) to understand properties of models.

  \item \textbf{Research}. Use all of the above to \emph{shape} and \emph{defend} our scientific opinions and make predictions and comparisons.
\end{enumerate}

\bigskip{}
How to use these fillable notes.
\begin{enumerate}
  \item Read it like a book! It is designed to be a workbook.
  \item Preview relevant pages before coming to lectures. It helps you focus in class and know what material is important. 
    \begin{itemize}
      \item Come to class with questions!
    \end{itemize}
  \item If an example is not discussed in class, then it is a do-at-home exercise. Do all do-at-home exercises.
  \item Review your notes, especially the boxed or starred paragraphs.
  \item Review examples to \hlmain{understand concepts} and perform calculations correctly.
\end{enumerate}

\end{document}
