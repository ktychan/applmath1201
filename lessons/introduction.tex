%! TeX program = lualatex
\documentclass[../main.tex]{subfiles}
\begin{document} \section{Introduction to mathematical models}
As scientists, we observe nature and try to create reasonable predictions of the physical world. Scientific theory describes the real world using models. Here are some familiar models.

\begin{enumerate}
  \item The solar system \url{https://eyes.nasa.gov/apps/solar-system/}
  \item Newton's \(F = ma\).
  \item The law of supply and demand in economics.
\end{enumerate}

Let's learn to ask basic questions about models (instead of taking them for granted because they are discovered by some famous person).
\begin{enumerate}[wide]
  \item Do they make sense? 

    \todo{leads to the discussion of everyday experience and dimensional homogeneity.}
    \blanklines{5}

  \item Are they accurate in every possible way? 

    \todo{leads to the discussion of making assumptions and curve fitting.}
    \blanklines{5}

  \item Do they provide knowledge and insights? 

    \todo{for fun and general interests.}
    \blanklines{5}

  \item Are they relatively easy to work with? 

    \todo{The purpose of this question is to project openness (from the instructor). The answer should be ``that depends, but different skills lead to different opportunities.''}
    \blanklines{5}
\end{enumerate}
\end{document}
