%! Tex Program = lualatex
\documentclass[t, 14pt]{beamer} 
\usetheme{metropolis} 

\everymath{\displaystyle} 
\setbeamersize{text margin left=.5cm} 
\setbeamersize{text margin right=.5cm} 
\beamertemplatenavigationsymbolsempty{}
\usepackage[normalem]{ulem}
\usepackage{fontawesome5}

\input{./colours.tex.preamble}
\input{./tikz.tex.preamble}
\setbeamertemplate{background}{
  \begin{tikzpicture}[overlay, remember picture]
    \node[teal!10, anchor=south] at (current page.south) {\resizebox{\textwidth}{!}{iClicker}};
  \end{tikzpicture}    
}

\newcommand{\yesno}{
 \begin{enumerate}
    \item Yes.
    \item No.
    \item 
    \item 
    \item I am not sure
  \end{enumerate} 
}

\usepackage{enumitem}
\setlist[enumerate]{label={\color{supp}(\alph*)}, itemsep={2ex}, topsep={1ex}}

\begin{document} 

\begin{frame}[c]
  \thispagestyle{empty}

  \begin{enumerate} 
    \item \underline{\hspace{4in}}
    \item \underline{\hspace{4in}}
    \item \underline{\hspace{4in}}
    \item \underline{\hspace{4in}}
    \item \underline{\hspace{4in}}
  \end{enumerate} 
\end{frame}

\section{Introduction week}

\begin{frame}[t]
  Office hours are Wednesdays and Fridays 2 pm to 3 pm in Math Help Centre.  I would like to set up one more office hour on Thursdays. Which of the following is most convenient for you on \emph{Thursdays}?

  \begin{enumerate} 
    \item 1 pm to 2:30 pm.
    \item 2 pm to 3:30 pm.
    \item 3 pm to 4:30 pm.
    \item 
    \item 
  \end{enumerate} 
\end{frame}


\begin{frame}[t]
  Suppose the log-log transformation of two quantities \(P\) versus \(Q\) is a straight line with slope \(0.5\).  Assume they exhibit either isometry or allometry. We must have that \(P\) and \(Q\) exhibit \underline{\hspace{1in}}.  

  \begin{enumerate} 
    \item isometry (but I am not certain).
    \item allometry (but I am not certain).
    \item isometry (I know for sure).
    \item allometry (I know for sure).
    \item I am not sure.
  \end{enumerate} 
\end{frame}

\section{Icebreaker}

\begin{frame}
  Is a hot dog a sandwich?
  
  \medskip
  \begin{enumerate} 
    \item Yes!
    \item Yes?
    \item No!
    \item No?
    \item I no longer know what a sandwich is.
  \end{enumerate} 
\end{frame}


\section{Recurrence}

\begin{frame}
  Formulate the question in the \faComment{} part of the Example~2.6.
\end{frame}

\begin{frame}
  What is a solution for the recurrence \(u_{t} = \frac{3}{2} u_{t}\)?
  \begin{enumerate}
    \item \(u_{t} = \left( \tfrac{3}{2} \right)^{t} u_{0}\).
    \item \(u_{t} = \left( \tfrac{3}{2} \right)t u_{0}\).
    \item \(u_{t} = \left( \tfrac{3}{2} \right)u_{0} + t\).
    \item \(u_{t} = \left( \tfrac{3}{2} \right) + t u_{0}\).
    \item I am not sure.
  \end{enumerate}
\end{frame}

\begin{frame}
  Which of the following are (seemingly) suitable models for Example~2.6.

  \begin{enumerate}
    \item \(u_{t} = \frac{1}{2} u_{t-1}\) for integer \(t \ge 1\) where the unit of \(t\) is years.
    \item \(u_{t} = \frac{1}{2} u_{t - 5730}\) for integer \(t \ge 1\) where the unit of \(t\) is years.
    \item \(u_{n} = \frac{1}{2} u_{n}\) for integer \(n \ge 1\) where the unit of \(n\) is multiples of \(5730\) years.
    \item 
    \item I am not sure. 
  \end{enumerate}
\end{frame}


\begin{frame}
  Why does \(\hat{b} = 2500\) make sense in the context of Example~2.5?
\end{frame}


\begin{frame}
  What is a suitable substitution for the recurrence \(p_{0} = 1\) and \(p_{t} = \tfrac{1}{2} p_{t-1} - \tfrac{1}{2}\) for integers \(t \ge 1\).

  \begin{enumerate} 
    \item \(p_{t} = u_{t} - 1\) and \(u_{0} = 2\).
    \item \(p_{t} = u_{t} + 1\) and \(u_{0} = 1\).
    \item \(p_{t} = u_{t} + \frac{1}{2}\) and \(u_{0} = 1/2\).
    \item \(p_{t} = u_{t} - \frac{1}{2}\) and \(u_{0} = 3/2\).
    \item I am not sure. 
  \end{enumerate}
\end{frame}

\section{Differential Equations}

\begin{frame}
  If \(k < 0\), then 

  \begin{enumerate} 
    \item \(P\) should increase or not change as \(t\) increases.
    \item \(P\) should not change as \(t\) increases.
    \item \(P\) should decrease or not change as \(t\) increases.
    \item 
    \item I am not sure. 
  \end{enumerate}
\end{frame}

\begin{frame}
  Which of the following functions are solutions to \(u'(t) = tu(t)\)?

  \begin{enumerate} 
    \item \(f_{1}(t) = e^{t^{2}}\).
    \item \(f_{2}(t) = 2 e^{t^{2}/2}\).
    \item \(f_{3}(t) = 3 t^{t^{2}/2}\).
    \item 
    \item I am not sure. 
  \end{enumerate}
\end{frame}

\begin{frame}
  What is your answer to Example~3.21?
  \begin{enumerate} 
    \item Find antiderivatives of \(x^{2}\).
    \item Integrate \(x^{2}\).
    \item Find antiderivative \(F\) of \(f(x) = x^{2}\) that satisfies \(F(0) = 4\).
    \item Find the tangent line of \(y = x^{2}\) passing through \((0,4)\).
    \item I am not sure. 
  \end{enumerate}
\end{frame}

\begin{frame}
  In Example~3.23, what should \(N'(t)\) be when solving for equilibria?

  \begin{enumerate} 
    \item \(N'(t)\) is some unknown constant.
    \item \(N'(t)\) should be \(0\), because an equilibrium is a constant function.
    \item There is not enough information to determine \(N'(t)\).
    \item 
    \item I am not sure. 
  \end{enumerate}
\end{frame}

\section{The logistic model}

\begin{frame}
  \small

  \begin{quote}
    ... the model of rabbit population \(N'(t) = (B-D) N(t)\) is pretty dumb if you take it too seriously at large values of \(N\) ... in the long run, this model predicts the existence of a ball of fifteen quitillion (\(15 \times 10^{18})\) rabbits expanding outward at half the speed of light \ldots{}
  \end{quote}

  An (slightly modified) excerpt from \emph{Modelling Life: The Mathematics of Biological Systems} (Garfinkel et al., 2017, p. 30).
\end{frame}

\begin{frame}
  In Example~3.30 part 1, how does the population grow according to the model?

  \begin{enumerate}
    \item \(N'(t) = \text{(an unknown constant)} N(t)\).
    \item \(N'(t) = \text{(a constant larger than \(r\))} N(t)\).
    \item \(N'(t) = \text{(a constant less than \(r\))} N(t)\).
    \item None of the above. 
    \item I am not sure.
  \end{enumerate}
\end{frame}

\section{Linear, first-order differential equations}

\begin{frame}
  What is the integrating factor for Example~4.4?

  \begin{enumerate}
    \item \(2/t\)
    \item \(2\ln|t|\)
    \item \(t^{2}\)
    \item \(e^{2\ln|t|}\)
    \item I am not sure. 
  \end{enumerate}
\end{frame}

\begin{frame}[c]
  Summarize the idea in Method~4.3 in your own words without any symbols or equations.
\end{frame}

\begin{frame}[c]
  Work on Example~4.8 for about \(10\) to \(15\) minutes.  When you are done, write down what you find challenging.
\end{frame}

\section{Mixing problems}

\begin{frame}
  Which of the following statements are true?

  \begin{enumerate} 
    \item The liquid outflow rate is constant.
    \item The liquid outflow rate changes with time.
    \item The green particles outflow rate is constant.
    \item The green particles outflow rate changes with time.
    \item I am not sure.
  \end{enumerate} 
\end{frame}

\begin{frame}[c]
  What is the outflow rate of change of sodium chloride in Example~4.16? Include units.
\end{frame}

\begin{frame}[c]
  What is the outflow rate of change of sodium chloride in Example~4.16?

  \begin{enumerate}
    \item \(10 \text{mL}/\text{sec}\) 
    \item \(u(t)/10 \text{mg}/\text{sec}\) 
    \item \(\frac{10}{10000} u(t) \text{mg}/\text{sec}\) 
    \item \(u(t) \text{mg}/\text{sec}\) 
    \item I am not sure.
  \end{enumerate}
\end{frame}

\begin{frame}[c]
  What if there is an additional condition that the rate of change of the substance is proportional to its mass with a proportionality constant \(k\)?

  \begin{enumerate}
    \item Add \(k u(t)\) to the right-hand side.
    \item Multiply the entire right-hand side by \(k\).
    \item No change.
    \item We need a second differential equation.
    \item I am not sure.
  \end{enumerate}
\end{frame}

\begin{frame}[t]
  If the drug enters the body at a constant rate (e.g., through an IV drip), does the constant \(k\) change?

  \begin{enumerate}
    \item Yes.
    \item No.
    \item 
    \item
    \item I am not sure. 
  \end{enumerate}
\end{frame}
\section{Bernoulli substitution}

\begin{frame}[t]
  Do we apply the Bernoulli substitution to the independent or the dependent variable of the differential equation?

  \begin{enumerate} 
    \item The dependent variable.
    \item The independent variable.
    \item 
    \item 
    \item I am not sure.
  \end{enumerate} 
\end{frame}

\begin{frame}[t]
  To which differential equation does the Bernoulli equation apply?

  \begin{enumerate} 
    \item \(v'(t) + 3v(t) = 1\).
    \item \(v'(t) + tv(t) = v^{2}(t)\).
    \item \(v'(t) + \sqrt{v(t)} = 1\).
    \item \(v(t) v'(t) + v(t) = e^{t}\).
    \item I am not sure.
  \end{enumerate} 
\end{frame}

\begin{frame}[t]
  Choose an appropriate substitution to transform
  \[
    s'(t) + (t-1)s(t)^{9} = s(t) t^{10}.
  \]

  \begin{enumerate} 
    \item Substitute \(u = t^{1-9}\).
    \item Substitute \(u = t^{1-10}\).
    \item Substitute \(u = s^{1-9}\).
    \item Substitute \(u = s^{1-10}\).
    \item I am not sure.
  \end{enumerate} 
\end{frame}

\begin{frame}[t]
  What is the Bernoulli substitution for the logistic equation?

  \begin{enumerate}
    \item \(u(t) = N(t)^{-1}\) and \(u'(t) = -N(t)^{-2}\).
    \item \(u(t) = N(t)^{-1}\) and \(u'(t) = -N(t)^{-2}N'(t)\).
    \item \(u(t) = N(t)^{2}\) and \(u'(t) = 2N(t)\).
    \item \(u(t) = N(t)^{-2}\) and \(u'(t) = -2 N(t)^{-3}N'(t)\).
    \item I am not sure.
  \end{enumerate}
\end{frame}




\section{Autonomous equations}

\begin{frame}[t]
  How many equilibria does \(y'(x) = (y(x) + 3)^{200} (3y(x) - 2) (1/3 - 5y(x))\) have?

  \begin{enumerate}
    \item None.
    \item \(3\)
    \item \(202\)
    \item
    \item I am not sure.
  \end{enumerate}
\end{frame}


\begin{frame}[c]
  Consider the differential equation \[u'(t) = u(t)(3 - u(t)/2)(2 - u(t))(u(t) - 5).\]

  Evaluate \(\lim_{t \to \infty} u(t)\) given \(u(0) = 1\).
\end{frame}

\section{Probability}

\begin{frame}[t]
  Dr. Wurst's hot dog test returns a \underline{\hspace{2cm}} for me.

  \begin{enumerate}
    \item true negative
    \item false positive
    \item true positive
    \item false negative
  \end{enumerate}

\end{frame}

\section{Misc}

\begin{frame}[t]
  What should we talk about this Friday, Feb 7?

  Pick one of the following. We will discuss the top 2 choices on Friday, 3 if time allows. 
  \begin{enumerate}
    \item Problem-solving strategies (when to use what technique).
    \item Modelling (metabolism or mixing problem with non-constant volume).
    \item Bernoulli substitution.
    \item Stability of equilibrium.
  \end{enumerate}
\end{frame}

\section{Probability}

\begin{frame}[c]
  Does it make sense to say ``the probability of winning the lottery is \(50\%\)?''

  Articulate your reasoning. 
\end{frame}

\begin{frame}[c]
  What is your answer for Example~6.11?

  \begin{enumerate}
    \item \(1/2\).
    \item \(1/8\).
    \item \(5/8\).
    \item None of the above.
    \item I am not sure.
  \end{enumerate}
\end{frame}

\begin{frame}[c]
  What is your answer for Example~6.12?

  \begin{enumerate}
    \item \(3/10\).
    \item \(4/10\).
    \item \(1/2\).
    \item None of the above.
    \item I am not sure.
  \end{enumerate}
\end{frame}

\begin{frame}
  If we know \(\mathbb{P}(E \cup F) = 1\), can we be sure that \(E \cup F\) is the entire sample space?

  \yesno
\end{frame}

\begin{frame}
  If we know \(\mathbb{P}(E \cap F) = 0\), can we be sure that \(E\) and \(F\) are mutually exclusive?

  \yesno
\end{frame}

\begin{frame}
  If we know \(\mathbb{P}(E_{1} \cap E_{2} \cap E_{3}) = 0\), can we be sure that \(E_{1}, E_{2}, E_{3}\) are mutually exclusive?

  \yesno
\end{frame}

\section{Random Variables}
\begin{frame}
  Does it make sense 
\end{frame}

\begin{frame}
  What is your favourite cupcake flavour?

  \begin{enumerate}
    \item Chocolate
    \item Lemon
    \item Red velvet
    \item Vanilla
    \item None of the above
  \end{enumerate}
\end{frame}
\end{document}
