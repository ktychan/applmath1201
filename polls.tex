%! Tex Program = lualatex
\documentclass[14pt]{beamer} 
\usetheme{metropolis} 
\everymath{\displaystyle} 
\setbeamersize{text margin left=.5cm} 
\setbeamersize{text margin right=.5cm} 
\beamertemplatenavigationsymbolsempty{}
\usepackage[normalem]{ulem}

\input{./colours.tex.preamble}
\input{./tikz.tex.preamble}
\setbeamertemplate{background}{
  \begin{tikzpicture}[overlay, remember picture]
    \node[teal!10, anchor=south] at (current page.south) {\resizebox{\textwidth}{!}{iClicker}};
  \end{tikzpicture}    
}

\begin{document} 

% \begin{frame}
%   Question
%
%   \medskip
%   \begin{itemize} \setlength\itemsep{2ex}
%     \item[(a)] 
%     \item[(b)] 
%     \item[(c)] 
%     \item[(d)] There is not enough information.
%     \item[(e)] I am not sure.
%   \end{itemize} 
% \end{frame}

\begin{frame}[c]
  \thispagestyle{empty}

  \begin{itemize} \setlength\itemsep{3ex}
    \item[(a)] \underline{\hspace{4in}}
    \item[(b)] \underline{\hspace{4in}}
    \item[(c)] \underline{\hspace{4in}}
    \item[(d)] \underline{\hspace{4in}}
    \item[(e)] \underline{\hspace{4in}}
  \end{itemize} 
\end{frame}

\section{Bernoulli substitution}

\begin{frame}[t]
  Do we apply the Bernoulli substitution to the independent or the dependent variable of the differential equation?

  \begin{itemize} \setlength\itemsep{1ex}
    \item[(a)] The dependent variable.
    \item[(b)] The independent variable.
    \item[(c)] 
    \item[(d)] 
    \item[(e)] I am not sure.
  \end{itemize} 
\end{frame}

\begin{frame}[t]
  Choose an appropriate substitution to transform
  \[
    s'(t) + (t-1)s(t)^{9} = s(t) t^{10}.
  \]

  \begin{itemize} \setlength\itemsep{1ex}
    \item[(a)] Substitute \(u = t^{1-n}\).
    \item[(b)] Substitute \(u = t^{n-1}\).
    \item[(c)] Substitute \(u = s^{1-n}\).
    \item[(d)] Substitute \(u = s^{n-1}\).
    \item[(e)] I am not sure.
  \end{itemize} 
\end{frame}

\section{Icebreaker}

\begin{frame}
  Is a hot dog a sandwich?
  
  \medskip
  \begin{itemize} \setlength\itemsep{2ex}
    \item[(a)] Yes!
    \item[(b)] Yes?
    \item[(c)] No!
    \item[(d)] No?
    \item[(e)] I no longer know what a sandwich is.
  \end{itemize} 
\end{frame}

\section{Recurrence techniques}

\begin{frame}
  Why does \(\hat{b} = 2500\) make sense in the context of Example~2.5?
\end{frame}


\begin{frame}[t]
  The following is a recurrence equation
  \[
    v_{0} = 1 \quad\text{ and }\quad v_{t} = 2 v_{t-1} - \frac{3}{4} \text{ for integers } t \ge 1.
  \]

  Which of the followings are solution(s) of the given recursion equation? Choose all that apply.

  \begin{enumerate}
    \item \(v_{t} = 2^{t}\)
    \item \(v_{t} = 2^{t} - \frac{3}{4}\)
    \item \(v_{t} = 2^{t-2} - \frac{3}{4}\)
    \item \(v_{t} = \frac{3}{4}\) if we change \(v_{0} = 3/4\)
  \end{enumerate}
\end{frame}

\end{document}

\begin{frame}
  Why does \(\hat{b} = 2500\) make sense in the context of Example~2.5?
\end{frame}


\begin{frame}[t]
  The following is a recurrence equation
  \[
    v_{0} = 1 \quad\text{ and }\quad v_{t} = 2 v_{t-1} - \frac{3}{4} \text{ for integers } t \ge 1.
  \]

  Which of the followings are solution(s) of the given recursion equation? Choose all that apply.

  \begin{enumerate}
    \item \(v_{t} = 2^{t}\)
    \item \(v_{t} = 2^{t} - \frac{3}{4}\)
    \item \(v_{t} = 2^{t-2} - \frac{3}{4}\)
    \item \(v_{t} = \frac{3}{4}\) if we change \(v_{0} = 3/4\)
  \end{enumerate}
\end{frame}

\begin{frame}
  \small

  \begin{quote}
    ... the model of rabbit population \(x' = bx\) is pretty dumb if you take it too seriously at large values of \(x\) ... in the long run, this model predicts that the existence of a ball of fifteen quitillion (\(15 \times 10^{18})\) rabbits expanding outward at half the speed of light \ldots{}
  \end{quote}

  An excerpt from \emph{Modelling Life: The Mathematics of Biological Systems} (Garfinkel et al., 2017, p. 30).
\end{frame}
\end{document}
