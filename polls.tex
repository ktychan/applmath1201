%! Tex Program = lualatex
\documentclass[14pt]{beamer} 
\usetheme{metropolis} 
\everymath{\displaystyle} 
\setbeamersize{text margin left=.5cm} 
\setbeamersize{text margin right=.5cm} 
\beamertemplatenavigationsymbolsempty{}
\usepackage[normalem]{ulem}
\usepackage{fontawesome5}

\input{./colours.tex.preamble}
\input{./tikz.tex.preamble}
\setbeamertemplate{background}{
  \begin{tikzpicture}[overlay, remember picture]
    \node[teal!10, anchor=south] at (current page.south) {\resizebox{\textwidth}{!}{iClicker}};
  \end{tikzpicture}    
}

\begin{document} 

% \begin{frame}
%   Question
%
%   \medskip
%   \begin{itemize} \setlength\itemsep{2ex}
%     \item[(a)] 
%     \item[(b)] 
%     \item[(c)] 
%     \item[(d)] There is not enough information.
%     \item[(e)] I am not sure.
%   \end{itemize} 
% \end{frame}

\begin{frame}[c]
  \thispagestyle{empty}

  \begin{itemize} \setlength\itemsep{3ex}
    \item[(a)] \underline{\hspace{4in}}
    \item[(b)] \underline{\hspace{4in}}
    \item[(c)] \underline{\hspace{4in}}
    \item[(d)] \underline{\hspace{4in}}
    \item[(e)] \underline{\hspace{4in}}
  \end{itemize} 
\end{frame}

\section{Introduction week}

\begin{frame}[t]
  Office hours are Wednesdays and Fridays 2 pm to 3 pm in Math Help Centre.  I would like to set up one more office hour on Thursday. Which of the following is most convenient for you on \emph{Thursdays}?

  \begin{itemize} \setlength\itemsep{1ex}
    \item[(a)] 1 pm to 2:30 pm.
    \item[(b)] 2 pm to 3:30 pm.
    \item[(c)] 3 pm to 4:30 pm.
    \item[(d)] 
    \item[(e)] 
  \end{itemize} 
\end{frame}


\begin{frame}[t]
  Suppose the log-log transformation of two quantities \(P\) versus \(Q\) is a straight line with slope \(0.5\).  Assume they exhibit either isometry or allometry. We must have that \(P\) and \(Q\) exhibit \underline{\hspace{1in}}.  

  \begin{itemize} \setlength\itemsep{1ex}
    \item[(a)] isometry (but I am not certain).
    \item[(b)] allometry (but I am not certain).
    \item[(c)] isometry (I know for sure).
    \item[(d)] allometry (I know for sure).
    \item[(e)] I am not sure.
  \end{itemize} 
\end{frame}

\section{Icebreaker}

\begin{frame}
  Is a hot dog a sandwich?
  
  \medskip
  \begin{itemize} \setlength\itemsep{2ex}
    \item[(a)] Yes!
    \item[(b)] Yes?
    \item[(c)] No!
    \item[(d)] No?
    \item[(e)] I no longer know what a sandwich is.
  \end{itemize} 
\end{frame}


\section{Recurrence techniques}

\begin{frame}
  Formulate the question in the \faComment{} part of the Example~2.6.
\end{frame}

\begin{frame}
  Which of the following are (seemingly) suitable models for Example~2.6.

  \begin{itemize} \setlength\itemsep{1ex}
    \item[(a)] \(u_{t} = \frac{1}{2} u_{t-1}\) for integer \(t \ge 1\) where the unit of \(t\) is years.
    \item[(b)] \(u_{t} = \frac{1}{2} u_{t - 5730}\) for integer \(t \ge 1\) where the unit of \(t\) is years.
    \item[(c)] \(u_{n} = \frac{1}{2} u_{n}\) for integer \(n \ge 1\) where the unit of \(n\) is multiples of \(5730\) years.
    \item[(d)] 
    \item[(e)] I am not sure. 
  \end{itemize}
\end{frame}


\begin{frame}
  Why does \(\hat{b} = 2500\) make sense in the context of Example~2.5?
\end{frame}


\begin{frame}
  What is a suitable substitution for the recurrence \(p_{0} = 1\) and \(p_{t} = \frac{1}{2} p_{t-1} - \frac{1}{2}\) for integers \(t \ge 1\).

  \begin{itemize} \setlength\itemsep{1ex}
    \item[(a)] \(p_{t} = u_{t} - 1\) and \(u_{0} = 2\).
    \item[(b)] \(p_{t} = u_{t} + 1\) and \(u_{0} = 1\).
    \item[(c)] \(p_{t} = u_{t} + \frac{1}{2}\) and \(u_{0} = 1/2\).
    \item[(d)] \(p_{t} = u_{t} - \frac{1}{2}\) and \(u_{0} = 3/2\).
    \item[(e)] I am not sure. 
  \end{itemize}
\end{frame}

\section{Differential Equations}

\begin{frame}
  In Example~3.23, what should \(N'(t)\) be when solving for equilibria?

  \begin{itemize} \setlength\itemsep{1ex}
    \item[(a)] \(N'(t)\) is some unknown constant.
    \item[(b)] \(N'(t)\) should be \(0\), because we are looking for equilibria.
    \item[(c)] There is not enough information to determine \(N'(t)\).
    \item[(d)] 
    \item[(e)] I am not sure. 
  \end{itemize}
\end{frame}


\section{The logistic model}

\begin{frame}
  \small

  \begin{quote}
    ... the model of rabbit population \(x' = bx\) is pretty dumb if you take it too seriously at large values of \(x\) ... in the long run, this model predicts that the existence of a ball of fifteen quitillion (\(15 \times 10^{18})\) rabbits expanding outward at half the speed of light \ldots{}
  \end{quote}

  An excerpt from \emph{Modelling Life: The Mathematics of Biological Systems} (Garfinkel et al., 2017, p. 30).
\end{frame}

\section{Bernoulli substitution}

\begin{frame}[t]
  Do we apply the Bernoulli substitution to the independent or the dependent variable of the differential equation?

  \begin{itemize} \setlength\itemsep{1ex}
    \item[(a)] The dependent variable.
    \item[(b)] The independent variable.
    \item[(c)] 
    \item[(d)] 
    \item[(e)] I am not sure.
  \end{itemize} 
\end{frame}

\begin{frame}[t]
  Choose an appropriate substitution to transform
  \[
    s'(t) + (t-1)s(t)^{9} = s(t) t^{10}.
  \]

  \begin{itemize} \setlength\itemsep{1ex}
    \item[(a)] Substitute \(u = t^{1-n}\).
    \item[(b)] Substitute \(u = t^{n-1}\).
    \item[(c)] Substitute \(u = s^{1-n}\).
    \item[(d)] Substitute \(u = s^{n-1}\).
    \item[(e)] I am not sure.
  \end{itemize} 
\end{frame}

\end{document}
